\begin{UseCase}{CU6}{Consultar registros de signos vitales}
	{	
		Este caso de uso permite al \hyperlink{actor:usuario}{Usuario} eliminar la información registrada de un paciente. 		 
	}
\UCccitem{Versión}{0.1}
\UCccsection{Administración de Requerimientos}
\UCccitem{Autor}{María Elsi Bernabé Aparicio}
%\UCccitem{Evaluador}{}
\UCccitem{Operación}{Editar}
\UCccitem{Prioridad}{Alta}
\UCccitem{Complejidad}{Media}
\UCccitem{Volatilidad}{Baja}
\UCccitem{Madurez}{Alta}
\UCccitem{Estatus}{Edición}
\UCccitem{Fecha del último estatus}{18 de octubre de 2018}

% Copie y pegue este bloque tantas veces como revisiones tenga el caso de uso.
% Esta sección la debe llenar solo el Revisor
%--------------------------------------------------------
%\UCccsection{Revisión Versión 0.1} % Anote la versión que se revisó.
%\UCccitem{Fecha}{}
%\UCccitem{Evaluador}{Elsi Bernabé Aparicio}
%\UCccitem{Resultado}{}
%\UCccitem{Observaciones}{}
%-------------------------------------------------------------------
	\UCsection{Atributos}
	\UCitem{Actor(es)}{\hyperlink{actor:usuario}{Usuario}.}
		\UCitem{Propósito}{Proporcionar un mecanismo que le permita al \hyperlink{actor:usuario}{Usuario} eliminar un paciente previamente registrado.}
	\UCitem{Entradas}{
		\begin{UClist}
			\UCli Ninguna.
		\end{UClist}	
	}
	\UCitem{Salidas}{
		\begin{UClist}
			\UCli \cdtIdRef{MSG1}{Operación exitosa}: Se muestra en la pantalla \cdtIdRef{IU1}{Consultar pacientes} indicando al actor que la información del paciente se eliminó exitosamente.
		\end{UClist}
		
	}
	\UCitem{Precondiciones}{
		\begin{UClist}
			\UCli Ninguna.
		\end{UClist}
	}
	
	\UCitem{Postcondiciones}{
		\begin{UClist}
			\UCli La información del paciente seleccionado será eliminada.
			\UCli El actor no podrá solicitar más mediciones de signos vitales para el paciente eliminado.
		\end{UClist}
	}

	\UCitem{Reglas de negocio}{
		\begin{UClist}
			\UCli Ninguna.
		\end{UClist}
	}
	
	\UCitem{Errores}{
		\begin{UClist}
			\UCli Ninguno.
		\end{UClist}
	}
	\UCitem{Tipo}{Terciario, extiende del caso de uso \cdtIdRef{CU3}{Consultar información del paciente}.}
\end{UseCase}

\begin{UCtrayectoria}
	\UCpaso[\UCactor] Da clic en el icono \btnEliminar{} de la pantalla \cdtIdRef{IU3}{Consultar información del paciente}.
	\UCpaso[\UCsist] \label{cu5:obtieneNombre}Obtiene el nombre del paciente que se requiere eliminar.
	\UCpaso[\UCsist] Muestra el mensaje \cdtIdRef{MSG6}{Eliminar elemento} en la pantalla \cdtIdRef{CU3}{Consultar información del paciente} con los siguientes parámetros: DETERMINADO ELEMENTO: {\em al paciente} y VALOR: nombre del paciente obtenido en el paso \ref{cu5:obtieneNombre} (¿Desea eliminar al paciente Carlos Granados?)
	\UCpaso[\UCactor] Selecciona la opción \cdtButton{Sí} del mensaje. \refTray{A}
	\UCpaso[\UCsist] Elimina la información registrada del paciente.
	\UCpaso[\UCsist] Muestra el mensaje \cdtIdRef{MSG1}{Operación exitosa} en la pantalla \cdtIdRef{IU1}{Consultar pacientes} indicando al actor que la eliminación se realizó correctamente.
\end{UCtrayectoria}


%-----------------------------------------------------------
\begin{UCtrayectoriaA}[Fin del caso de uso]{A}{El actor no confirma la eliminación del paciente.}
	\UCpaso[\UCactor] Selecciona la opción \cdtButton{No} del mensaje.
	\UCpaso[\UCsist] Muestra la pantalla \cdtIdRef{CU3}{Consultar información del paciente}.
\end{UCtrayectoriaA}
