\begin{UseCase}{CU2}{Registrar paciente}
	{	
		Permite al \hyperlink{actor:usuario}{Usuario} de la aplicación móvil, registrar un nuevo paciente del cual requiere monitorear los signos vitales de temperatura y frecuencia cardíaca.\\
		
		Para registrar un nuevo paciente, el usuario debe ingresar los datos personales del mismo, así como en número telefónico asignado al módulo GSM del sistema que utilizará el paciente.
		 
		 
	}
\UCccitem{Versión}{0.1}
\UCccsection{Administración de Requerimientos}
\UCccitem{Autor}{María Elsi Bernabé Aparicio}
%\UCccitem{Evaluador}{}
\UCccitem{Operación}{Registro}
\UCccitem{Prioridad}{Alta}
\UCccitem{Complejidad}{Media}
\UCccitem{Volatilidad}{Baja}
\UCccitem{Madurez}{Alta}
\UCccitem{Estatus}{Edición}
\UCccitem{Fecha del último estatus}{18 de octubre de 2018}

% Copie y pegue este bloque tantas veces como revisiones tenga el caso de uso.
% Esta sección la debe llenar solo el Revisor
%--------------------------------------------------------
%\UCccsection{Revisión Versión 0.1} % Anote la versión que se revisó.
%\UCccitem{Fecha}{}
%\UCccitem{Evaluador}{Elsi Bernabé Aparicio}
%\UCccitem{Resultado}{}
%\UCccitem{Observaciones}{}
%-------------------------------------------------------------------
	\UCsection{Atributos}
	\UCitem{Actor(es)}{\hyperlink{actor:usuario}{Usuario}}
	\UCitem{Entradas}{
		\begin{UClist}
			\UCli Nombre completo del paciente que usará el sistema. \ioEscribir
			\UCli Número telefónico de la tarjeta SIM insertada en el módulo GSM del sistema. \ioEscribir
			\UCli Fecha de nacimiento del paciente. \ioCalendario
			\UCli Sexo del paciente. \ioSeleccionar
		\end{UClist}	
	}
	\UCitem{Salidas}{
		\begin{UClist}
			\UCli 
		\end{UClist}
		
	}
	\UCitem{Precondiciones}{
		\begin{UClist}
			\UCli Ninguna.
		\end{UClist}
	}
	
	\UCitem{Postcondiciones}{
		\begin{UClist}
			\UCli Se registrarán los datos del paciente en la aplicación móvil.
			\UCli El usuario podrá consultar los datos del paciente.
			\UCli El usuario podrá editar los datos registrados del paciente.
			\UCli El usuario podrá eliminar al paciente registrado.
			\UCli El usuario podrá solicitar la medición de los signos vitales del paciente.
		\end{UClist}
	}

	\UCitem{Reglas de negocio}{
		\begin{UClist}
			\UCli Ninguno.
		\end{UClist}
	}
	
	\UCitem{Errores}{
		\begin{UClist}
			\UCli Ninguno.
		\end{UClist}
	}
	\UCitem{Tipo}{Primario}
\end{UseCase}

\begin{UCtrayectoria}
	\UCpaso[\UCactor] Selecciona la opción \btnRegistro{} de la pantalla \cdtIdRef{IU1}{Consultar pacientes}.
	\UCpaso[\UCsist] Obtiene los datos del catálogo sexo.
	\UCpaso[\UCsist] Muestra la pantalla \cdtIdRef{IU2}{Registrar paciente}.
	\UCpaso[\UCactor] Ingresa el nombre del paciente.
	\UCpaso[\UCactor] Ingresa el número telefónico del paciente.
	\UCpaso[\UCactor] Selecciona la fecha de nacimiento del paciente.
	\UCpaso[\UCactor] Selecciona el sexo del paciente.
	\UCpaso[\UCactor] Da clic en el icono \btnRegistrar{} de la pantalla \cdtIdRef{IU2}{Registrar paciente}.
	\UCpaso[\UCsist] Verifica que no se omitan los datos requeridos con base en la regla de negocio \cdtIdRef{RN1}{Campos obligatorios}. \refTray{A}
	\UCpaso[\UCsist] Verifica que el formato de los datos ingresados sea válido con base en la regla de negocio \cdtIdRef{RN3}{Información correcta}. \refTray{B}
	\UCpaso[\UCsist] Verifica que no exista un paciente registrado con el mismo número de teléfono. \refTray{C}
	\UCpaso[\UCsist] Verifica que la fecha de nacimiento del paciente sea válida como lo especifica la regla de negocio \cdtIdRef{RN3}{Fecha de nacimiento válida}. \refTray{D}
	\UCpaso[\UCsist] Calcula la fecha de nacimiento del paciente con base en la fecha de nacimiento registrada.
	\UCpaso[\UCsist] Registra la información ingresada y calculada del paciente.
	\UCpaso[\UCsist] Muestra el mensaje \cdtIdRef{MSG1}{Operación exitosa} indicando al actor que el registro se realizó correctamente.
\end{UCtrayectoria}

%\begin{UCtrayectoriaA}{A}{.}
%	
%\end{UCtrayectoriaA}

