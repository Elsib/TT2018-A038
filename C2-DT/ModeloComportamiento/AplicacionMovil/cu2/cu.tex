\begin{UseCase}{CU2}{Registrar paciente}
	{	
		Permite al \hyperlink{actor:usuario}{Usuario} de la aplicación móvil, registrar un nuevo paciente del cual requiere monitorear los signos vitales de temperatura y frecuencia cardíaca.\\
		
		Para registrar un nuevo paciente, el usuario debe ingresar los datos personales del mismo, así como en número telefónico asignado al módulo 4G del sistema que utilizará el paciente.
		 
		 
	}
\UCccitem{Versión}{0.1}
\UCccsection{Administración de Requerimientos}
\UCccitem{Autor}{María Elsi Bernabé Aparicio}
%\UCccitem{Evaluador}{}
\UCccitem{Operación}{Registro}
\UCccitem{Prioridad}{Alta}
\UCccitem{Complejidad}{Media}
\UCccitem{Volatilidad}{Baja}
\UCccitem{Madurez}{Alta}
\UCccitem{Estatus}{Edición}
\UCccitem{Fecha del último estatus}{18 de octubre de 2018}

% Copie y pegue este bloque tantas veces como revisiones tenga el caso de uso.
% Esta sección la debe llenar solo el Revisor
%--------------------------------------------------------
%\UCccsection{Revisión Versión 0.1} % Anote la versión que se revisó.
%\UCccitem{Fecha}{}
%\UCccitem{Evaluador}{Elsi Bernabé Aparicio}
%\UCccitem{Resultado}{}
%\UCccitem{Observaciones}{}
%-------------------------------------------------------------------
	\UCsection{Atributos}
	\UCitem{Actor(es)}{\hyperlink{actor:usuario}{Usuario}.}
	\UCitem{Propósito}{Proporcionar un mecanismo que le permita al \hyperlink{actor:usuario}{Usuario} registrar un paciente.}
	\UCitem{Entradas}{
		\begin{UClist}
			\UCli Nombre completo del paciente que usará el sistema. \ioEscribir
			\UCli Número telefónico de la tarjeta SIM insertada en el módulo GSM del sistema. \ioEscribir
			\UCli Fecha de nacimiento del paciente. \ioCalendario
			\UCli Sexo del paciente. \ioSeleccionar
		\end{UClist}	
	}
	\UCitem{Salidas}{
		\begin{UClist}
			\UCli \cdtIdRef{MSG1}{Operación exitosa}: Se muestra en la pantalla \cdtIdRef{IU1}{Consultar pacientes} indicando al actor que el registro del paciente se realizó exitosamente.
		\end{UClist}
		
	}
	\UCitem{Precondiciones}{
		\begin{UClist}
			\UCli Ninguna.
		\end{UClist}
	}
	
	\UCitem{Postcondiciones}{
		\begin{UClist}
			\UCli Se registrarán los datos del paciente en la aplicación móvil.
			\UCli El usuario podrá consultar los datos del paciente.
			\UCli El usuario podrá editar los datos registrados del paciente.
			\UCli El usuario podrá eliminar al paciente registrado.
			\UCli El usuario podrá solicitar la medición de los signos vitales del paciente.
		\end{UClist}
	}

	\UCitem{Reglas de negocio}{
		\begin{UClist}
			\UCli \cdtIdRef{RN1}{Campos obligatorios}: Verifica que todos los campos obligatorios hayan sido ingresados.
			\UCli \cdtIdRef{RN2}{Información correcta} Verifica que el formato de los campos ingresados cumpla con el tipo de dato definido.
			\UCli \cdtIdRef{RN3}{Fecha de nacimiento válida}: Verifica que la fecha de nacimiento registrada se encuentre dentro dentro del rango considerado válido.
		\end{UClist}
	}
	
	\UCitem{Errores}{
		\begin{UClist}
			\UCli \cdtIdRef{MSG2}{Falta dato obligatorio}: Se muestra en la pantalla \cdtIdRef{IU2}{Registrar paciente} indicando al actor que omitió alguno de los campos obligatorios.
			\UCli \cdtIdRef{MSG3}{Formato de campo incorrecto}: Se muestra en la pantalla \cdtIdRef{IU2}{Registrar paciente} indicando al actor que alguno de los campos ingresados no cumple con el formato definido.
			\UCli \cdtIdRef{MSG4}{Número telefónico previamente registrado}: Se muestra en la pantalla \cdtIdRef{IU2}{Registrar paciente} indicando al actor que el número telefónico registrado ya se encuentra asociado a otro paciente.
			\UCli \cdtIdRef{MSG5}{Fecha de nacimiento inválida}: Se muestra en la pantalla \cdtIdRef{IU2}{Registrar paciente} indicando al actor que la fecha de nacimiento ingresada se encuentra fuera del rango válido.
		\end{UClist}
	}
	\UCitem{Tipo}{Primario}
\end{UseCase}

\begin{UCtrayectoria}
	\UCpaso[\UCactor] Selecciona la opción \btnRegistro{} de la pantalla \cdtIdRef{IU1}{Consultar pacientes}.
	\UCpaso[\UCsist] Obtiene los datos del catálogo sexo.
	\UCpaso[\UCsist] Muestra la pantalla \cdtIdRef{IU2}{Registrar paciente}.
	\UCpaso[\UCactor] \label{cuir2:ingresaNombre}Ingresa el nombre del paciente.
	\UCpaso[\UCactor] Ingresa el número telefónico del paciente.
	\UCpaso[\UCactor] Selecciona la fecha de nacimiento del paciente.
	\UCpaso[\UCactor] Selecciona el sexo del paciente.
	\UCpaso[\UCactor] Da clic en el icono \btnRegistrar{} de la pantalla \cdtIdRef{IU2}{Registrar paciente}.
	\UCpaso[\UCsist] Verifica que no se omitan los datos requeridos con base en la regla de negocio \cdtIdRef{RN1}{Campos obligatorios}. \refTray{A}
	\UCpaso[\UCsist] Verifica que el formato de los datos ingresados sea válido con base en la regla de negocio \cdtIdRef{RN2}{Información correcta}. \refTray{B}
	\UCpaso[\UCsist] Verifica que no exista un paciente registrado con el mismo número de teléfono. \refTray{C}
	\UCpaso[\UCsist] Verifica que la fecha de nacimiento del paciente sea válida como lo especifica la regla de negocio \cdtIdRef{RN3}{Fecha de nacimiento válida}. \refTray{D}
	\UCpaso[\UCsist] Registra la información ingresada del paciente.
	\UCpaso[\UCsist] Muestra el mensaje \cdtIdRef{MSG1}{Operación exitosa} en la pantalla \cdtIdRef{IU1}{Consultar pacientes} indicando al actor que el registro se realizó correctamente.
\end{UCtrayectoria}


%-----------------------------------------------------------
\begin{UCtrayectoriaA}{A}{El actor omitió alguno de los campos obligatorios.}
	\UCpaso[\UCsist] Muestra el mensaje \cdtIdRef{MSG2}{Falta dato obligatorio} en la pantalla \cdtIdRef{IU2}{Registrar paciente}.
	\UCpaso[] Regresa al paso \ref{cuir2:ingresaNombre} de la trayectoria principal.
\end{UCtrayectoriaA}

%-----------------------------------------------------------
\begin{UCtrayectoriaA}{B}{El actor ingresó algún dato que no cumple con el tipo de dato definido.}
	\UCpaso[\UCsist] Muestra el mensaje \cdtIdRef{MSG3}{Formato de campo incorrecto} en la pantalla \cdtIdRef{IU2}{Registrar paciente}.
	\UCpaso[] Regresa al paso \ref{cuir2:ingresaNombre} de la trayectoria principal.
\end{UCtrayectoriaA}

%-----------------------------------------------------------
\begin{UCtrayectoriaA}{C}{El actor ingresó el número telefónico de un paciente que ya existe.}
	\UCpaso[\UCsist] Muestra el mensaje \cdtIdRef{MSG4}{Número telefónico previamente registrado} en la pantalla \cdtIdRef{IU2}{Registrar paciente}.
	\UCpaso[] Regresa al paso \ref{cuir2:ingresaNombre} de la trayectoria principal.
\end{UCtrayectoriaA}

%-----------------------------------------------------------
\begin{UCtrayectoriaA}{D}{El actor ingresó una fecha de nacimiento fuera del rango considerado válido.}
	\UCpaso[\UCsist] Muestra el mensaje \cdtIdRef{MSG5}{Fecha de nacimiento inválida} en la pantalla \cdtIdRef{IU2}{Registrar paciente}.
	\UCpaso[] Regresa al paso \ref{cuir2:ingresaNombre} de la trayectoria principal.
\end{UCtrayectoriaA}