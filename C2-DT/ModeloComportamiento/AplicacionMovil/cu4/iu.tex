\subsection{IU4 Editar información del paciente}

\subsubsection{Objetivo}
	
Esta pantalla permite al \hyperlink{actor:usuario}{Usuario} de la aplicación móvil, modificar la información previamente registrada para un paciente, esto con el fin de mantener actualizada su información en caso de que algún dato haya sido registrado de forma errónea o haya sido actualizado.

\subsubsection{Diseño}
En la figura \ref{IU4} se muestra la pantalla ''Editar información del paciente'', por medio de la cual el usuario podrá modificar los datos personales y número de teléfono de un paciente registrado anteriormente.\\

Al igual que en el registro, se validará que se ingresen todos los datos necesarios, los cuales son:
\begin{enumerate}
	\item Nombre del paciente que será monitoreado.
	\item Número telefónico de la tarjeta SIM que tendrá integrada el módulo de comunicación GSM para en envío de mensajes.
	\item Fecha de nacimiento del paciente, la cual debe encontrarse dentro del rango establecido con base en la regla de negocio \cdtIdRef{RN3}{Fecha de nacimiento válida}.
	\item Sexo del paciente.
\end{enumerate}

Una vez validados todos los datos del paciente, éste será actualizado y podrá ser visualizado en la lista de pacientes de la pantalla \cdtIdRef{IU1}{Consultar pacientes}.

    
 \IUfig[.6]{../ModeloComportamiento/AplicacionMovil/cu4/iu4.png}{IU4}{Editar información del paciente}

\subsubsection{Comandos}
	\begin{itemize}
		\item \btnRegistrar{} [Editar paciente]: Permite al actor confirmar los datos ingresados del paciente para finalizar su edición. Dirige a la pantalla \cdtIdRef{IU1}{Consultar pacientes}.
	\end{itemize}
\clearpage