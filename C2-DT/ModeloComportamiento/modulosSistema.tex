
%=========================================================
\section{Módulos del sistema}

El sistema se encuentra organizado en módulos, con la finalidad de agrupar y administrar de mejor manera los requerimientos funcionales del sistema. Dividir el sistema en módulos permite visualizar e identificar rápidamente aquellos aspectos funcionales que pueden tratarse conjuntamente. La figura \ref{fig:SubsistemasSAE} muestra los módulos propuestos para el sistema.

%\begin{figure}[h!]
%	\begin{center}
%		\includegraphics[scale=0.4]{ModeloComportamiento/images/modulosSAEposgrado.png}
%		\caption{Módulos del \sae.}
%		\label{fig:SubsistemasSAE}
%	\end{center}
%\end{figure}

A continuación se describen de manera general cada uno de los módulos:

\begin{itemize}
	\item {\bf Aplicación Móvil:}
	
	\item {\bf Sistema Embebido:} 
	
	%\item {\bf Sistema Embebido:} Agrupa los casos de uso que tienen que ver con el proceso de gestión académica en el departamento de Posgrado de la Escuela Libre de Derecho, los cuales definen la oferta educativa, los grupos, profesores y horarios en los que se impartirán cada uno de los programas académicos del departamento de posgrado de la Escuale Libre de Derecho.
	
\end{itemize}
\pagebreak