\begin{UseCase}{CU7}{Enviar mediciones}
	{	

		 Este casos de uso permite al sistema enviar un informe de los valores de los signos vitales de un paciente mediante un mensaje SMS a al número celular de la persona interesada en el monitoreo. El envió de las mediciones puede realizarse en dos momentos: 
		 \begin{enumerate}
		 	\item Cuando se sobrepasa uno de los umbrales establecidos para el valor de temperatura o frecuencia cardíaca.
		 	\item Cuando se cumple el tiempo definido de envío de muestras.
		 \end{enumerate}
		 
	}
\UCccitem{Versión}{0.1}
\UCccsection{Administración de Requerimientos}
\UCccitem{Autor}{María Elsi Bernabé Aparicio}
\UCccitem{Evaluador}{}
\UCccitem{Operación}{}
\UCccitem{Prioridad}{Alta}
\UCccitem{Complejidad}{Alta}
\UCccitem{Volatilidad}{Media}
\UCccitem{Madurez}{Media}
\UCccitem{Estatus}{Edición}
\UCccitem{Fecha del último estatus}{}

% Copie y pegue este bloque tantas veces como revisiones tenga el caso de uso.
% Esta sección la debe llenar solo el Revisor
%--------------------------------------------------------
%\UCccsection{Revisión Versión 0.1} % Anote la versión que se revisó.
%\UCccitem{Fecha}{}
%\UCccitem{Evaluador}{Elsi Bernabé Aparicio}
%\UCccitem{Resultado}{}
%\UCccitem{Observaciones}{}
%-------------------------------------------------------------------
	\UCsection{Atributos}
	\UCitem{Actor(es)}{\cdtRef{Actor:sistema}{Sistema}}
	\UCitem{Entradas}{
		\begin{UClist}
			\UCli Valores analógicos de frecuencia y temperatura del paciente.
		\end{UClist}	
	}
	\UCitem{Salidas}{
		\begin{UClist}
			\UCli Valor de medición de temperatura y frecuencia cardíaca del paciente.
			\UCli Fecha y hora de envío del mensaje SMS.
			\UCli Mensaje SMS con la información obtenida.
		\end{UClist}
		
	}
	\UCitem{Precondiciones}{
		\begin{UClist}
			\UCli Que se haya cumplido el tiempo definido de envío de mensaje con mediciones.
			\UCli Que uno de los valores de temperatura o frecuencia cardíaca se encuentre fuera de los umbrales establecidos.
		\end{UClist}
	}
	
	\UCitem{Postcondiciones}{
		\begin{UClist}
			\UCli ...
		\end{UClist}
	}

	\UCitem{Reglas de negocio}{
		\begin{UClist}
			\UCli Ninguno.
		\end{UClist}
	}
	
	\UCitem{Errores}{
		\begin{UClist}
			\UCli Ninguno.
		\end{UClist}
	}
	\UCitem{Tipo}{}
\end{UseCase}

\begin{UCtrayectoria}
	\UCpaso[\UCactor] Selecciona la opción 
\end{UCtrayectoria}

\begin{UCtrayectoriaA}{A}{.}
	
\end{UCtrayectoriaA}

