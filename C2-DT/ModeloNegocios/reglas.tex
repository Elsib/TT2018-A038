% El tipo de regla de negocio (tercer parámetro del entorno 'BusinessRule') se describe en la siguiente tabla:
%---------------------------------------------------------------------------------------------------------------,
% TIPOS		|		DEFINICION		|	EJEMPLO						|
%---------------|---------------------------------------|-------------------------------------------------------|
% Habilitador   | La sentencia habilita o restringe 	| * Se pueden recibir solicitudes del tipo A, B y C.	|
%		| hacer algo o  una funcionalidad.	| * Se permite hacer algo si se tiene el estado X.	|
%---------------|---------------------------------------|-------------------------------------------------------|
% Cronometrado	| Se permite de manera controlada 	| * Se permiten hasta dos solicitudes del tipo D	| 
%		| por un contador.			|   por persona.					|
% 		|					| * El acceso al sistema se permite si no se tiene 	|
%  		|					|   más de X número de intentos fallidos.		|
%---------------|---------------------------------------|-------------------------------------------------------|
% Ejecutivo	| Autorizado por un superior, un perfil | * Se permite registrar extemporaneamente si lo 	|
%		| particular debe autorizar.		|   autoriza X.						|
%---------------|---------------------------------------|-------------------------------------------------------|
% Derivación	| Son de cálculo e inferencia, 		| * Un alumno irregular es aquel que tiene las 		|
%		| es un cálculo o conclusión derivados 	|   siguientes cacteristicas: A, B, C. 			|
%		| de un conjunto de datos. Puede ser una| * El formato de un correo o CURP.			|
%		| fórmula que dice cómo calcular algo 	|							|
%		| o el formato de un dato.		|							|
%---------------|---------------------------------------|-------------------------------------------------------|
% Restricción	| Restringe una funcionalidad o relación| * Traslape de fechas o periodos empalmados.		|
% 		| entre dos o mas objetos.		|							|
%  		|					|							|
%---------------------------------------------------------------------------------------------------------------'

% No editar las reglas cuyo estatus es APROBADO.

\section{Reglas de negocio}

\subsection{Reglas derivadas del sistema}
%------------------------------------------------------------------------------------------------------------------
\begin{BusinessRule}{RN-S1}{Información correcta}
	{Restricción}
	{Controla la operación}
	\BRitem{Versión}{1.0}
	\BRitem{Autor}{Victor Lozano Ortega}
	\BRitem{Estatus}{Terminado}
	\BRitem{Descripción}{Los datos proporcionados al sistema que son marcados como ``requeridos'' no se deben omitir. 
		Todos los datos proporcionados al sistema deben respetar el formato y pertenecer al \cdtRef{gls:tipoDato}{tipo de dato} 
		especificado en el modelo de información; así como estar dentro de la longitud máxima o mínima definida en el diccionario de datos}
	\BRitem{Referenciado por}{
	%CUR
	\cdtIdRef{CUR 1}{Iniciar sesión},
	\cdtIdRef{CUR 2}{Recuperar contraseña},
	\cdtIdRef{CUR 3}{Solicitar inscripción},
	\cdtIdRef{CUR 4}{Registrar coordinador del programa},
	\cdtIdRef{CUR 6}{Completar información escolar},
	\cdtIdRef{CUR 7}{Modificar información escolar},
	\cdtIdRef{CUR 10}{Registrar responsable del programa},
	\cdtIdRef{CUR 11}{Modificar responsable del programa},
	\cdtIdRef{CUR 14}{Registrar integrante de línea de acción},
	\cdtIdRef{CUR 15}{Modificar integrante de línea de acción},
	\cdtIdRef{CUIBA 2}{Registrar información base para indicadores de agua},
	\cdtIdRef{CUIBB 2}{Registrar información base para indicadores de biodiversidad},
	\cdtIdRef{CUIBB 4}{Registrar fauna},
	\cdtIdRef{CUIBB 7}{Registrar flora},
	\cdtIdRef{CUIBR 2}{Registrar residuo sólido},
	\cdtIdRef{CUIBR 3}{Modificar residuo sólido},
	\cdtIdRef{CUIBCR 2}{Registrar información base para indicadores de consumo responsable},
	\cdtIdRef{CUIBE 2}{Registrar información base para indicadores de energía},
	\cdtIdRef{CUP 2}{Registrar objetivo},
	\cdtIdRef{CUP 3}{Modificar objetivo},
	\cdtIdRef{CUP 8}{Registrar acción},
	\cdtIdRef{CUP 9}{Modificar acción},
	\cdtIdRef{CUPAE 1}{Registrar meta de ambiente escolar},
	\cdtIdRef{CUPAE 2}{Modificar meta de ambiente escolar},
	\cdtIdRef{CUPB 1}{Registrar meta de biodiversidad},
	\cdtIdRef{CUPB 2}{Modificar meta de biodiversidad},
	\cdtIdRef{CUPCR 1}{Registrar meta de consumo responsable},
	\cdtIdRef{CUPCR 2}{Modificar meta de consumo responsable},
	\cdtIdRef{CUPRS 1}{Registrar meta de residuo sólido},
	\cdtIdRef{CUPRS 2}{Modificar meta de residuo sólido},
	\cdtIdRef{CUPR 1}{Registrar residuo sólido del plan de acción},
	\cdtIdRef{CUPR 2}{Modificar residuo sólido del plan de acción},
	\cdtIdRef{CUPL 1}{Registrar meta},
	\cdtIdRef{CUPL 2}{Modificar meta},
	\cdtIdRef{CUS 13}{Registrar fauna},
	\cdtIdRef{CUS 16}{Registrar flora},
	\cdtIdRef{CUS 19}{Registrar avances de acciones de agua},
	\cdtIdRef{CUS 20}{Registrar avances de meta de agua},
	\cdtIdRef{CUS 21}{Actualizar información del consumo de agua},
	\cdtIdRef{CUS 23}{Registrar avances de acciones de energía},
	\cdtIdRef{CUS 24}{Registrar avances de meta de energía},
	\cdtIdRef{CUS 25}{Actualizar información del consumo de energía},
	\cdtIdRef{CUS 27}{Registrar avances de acciones de residuo sólido},
	\cdtIdRef{CUS 28}{Registrar avances de meta de residuo sólido},
	\cdtIdRef{CUS 30}{Registrar residuo sólido},
	\cdtIdRef{CUS 31}{Modificar residuo sólido},}
	%\cdtIdRef{CUR20}{Aprobar solicitud de inscripción}}
\end{BusinessRule}
%------------------------------------------------------------------------------------------------------------------
\begin{BusinessRule}{RN-S2}{Formato de correo electrónico}
    {Derivación}
    {Controla la operación}
    \BRitem{Versión}{1.0}
    \BRitem{Autor}{Victor Lozano Ortega}
    \BRitem{Estatus}{Terminado}
    \item[Descripción:]El correo electrónico debe ser una cadena de caracteres con la siguiente estructura ordenada:
    \begin{enumerate}
	\item Cadena de caracteres.
	\item Carácter ``@''.
	\item Cadena de caracteres.
	\item Carácter ``.''.
	\item Cadena de caracteres.
    \end{enumerate}
    \item[Ejemplo:]{correo@dominio.com}
    \BRitem{Referenciado por}{
    \cdtIdRef{CUR 4}{Registrar coordinador del programa},
    \cdtIdRef{CUR 10}{Registrar responsable del programa},
    \cdtIdRef{CUR 11}{Modificar responsable del programa},
    \cdtIdRef{CUR 14}{Registrar integrante de línea de acción},
    \cdtIdRef{CUR 15}{Modificar integrante de línea de acción}}
\end{BusinessRule}
%------------------------------------------------------------------------------------------------------------------
\begin{BusinessRule}{RN-S3}{Formato de la contraseña}
    {Derivación}
    {Controla la operación}
    \BRitem{Versión}{1.0}
    \BRitem{Autor}{Victor Lozano Ortega}
    \BRitem{Estatus}{Terminado}
    \BRitem{Descripción}{La longitud de la contraseña debe ser entre 8 y 16 caracteres, debe contener al menos una letra en mayúscula, 
	un dígito y no debe contener la subcadena ``nombre propio del actor'' en ella. Se deberán considerar las siguientes restricciones: 
	No debe haber caracteres especiales, no debe haber letras acentuadas y no deben existir espacios}
    \BRitem{Ejemplo}{Bond007}
    \BRitem{Referenciado por}{\cdtIdRef{CUR 19}{Verificar correo electrónico}}%PENDIENTE
\end{BusinessRule}

%------------------------------------------------------------------------------------------------------------------
\begin{BusinessRule}{RN-S4}{Unicidad de identificadores}
	{Restricción}
	{Controla la operación}
	\BRitem{Versión}{1.0}
	\BRitem{Autor}{Victor Lozano Ortega}
	\BRitem{Estatus}{Terminado}
	\BRitem{Descripción}{En el conjunto de entidades del sistema, la \cdtRef{escuela:cct}{clave de centro de trabajo} y el \cdtRef{cuenta:usuario}{nombre de usuario} deben ser únicos}
	\item[Ejemplo de cumplimiento:] El conjunto de escuelas:
		\begin{Citemize}
			\item (Nombre de la escuela: ``Antonio Camacho Salazar'', Clave de centro de trabajo: ``15DPR2497K'')
			\item (Nombre de la escuela: ``Coatlicue'', Clave de centro de trabajo: ``27DPR4279H'')
			\item (Nombre de la escuela: ``Antonio Camacho Salazar'', Clave de centro de trabajo: ``79DPR4251H'')
		\end{Citemize}
		Es correcto debido a que las escuelas tienen una clave de centro de trabajo diferente.
		\item[Ejemplo de fallo:] El conjunto de escuelas:    
		\begin{Citemize}
			\item (Nombre de la escuela: ``Antonio Camacho Salazar'', Clave de centro de trabajo: ``15DPR2497K'')
			\item (Nombre de la escuela: ``Coatlicue'', Clave de centro de trabajo: ``15DPR2497K'')
			\item (Nombre de la escuela: ``Antonio Camacho Salazar'', Clave de centro de trabajo: ``79DPR4251H'')
		\end{Citemize}
		Es {\bf \underline{incorrecto}} debido a que hay dos escuelas con la misma clave de centro de trabajo.
	\BRitem{Referenciado por}{
	\cdtIdRef{CUR 3}{Solicitar inscripción},}
\end{BusinessRule}

%------------------------------------------------------------------------------------------------------------------
\begin{BusinessRule}{RN-S5}{Archivos permitidos en el sistema}
    {Derivación}
    {Controla la operación}
    \BRitem{Versión}{1.0}
    \BRitem{Autor}{Natalia Giselle Hernández Sánchez}
    \BRitem{Estatus}{Terminado}
    \BRitem{Descripción}{Los archivos permitidos en el sistema son con formato ``PDF'' y tamaño máximo de 2 MB} %El tamaño es una propuesta
    \BRitem{Referenciado por}{
    \cdtIdRef{CUR 4}{Registrar coordinador del programa}}
\end{BusinessRule}
%------------------------------------------------------------------------------------------------------------------
\begin{BusinessRule}{RN-S6}{Títulos de las administraciones}
    {Derivación}
    {Controla la operación}
    \BRitem{Versión}{1.0}
    \BRitem{Autor}{Natalia Giselle Hernández Sánchez}
    \BRitem{Estatus}{Terminado}
    \BRitem{Descripción}{Los títulos de la administración de metas y la administración de acciones especificarán la línea de acción a la que están asociadas, p.e. la administración de metas de la línea de acción ``Energía'' tendrá
    como título ``Administrar metas {\em de energía}''} 
    \BRitem{Referenciado por}{
    \cdtIdRef{CUP 5}{Administrar metas},
    \cdtIdRef{CUP 7}{Administrar acciones}}
\end{BusinessRule}

% %------------------------------------------------------------------------------------------------------------------
% \begin{BusinessRule}{RN-S7}{Información el calculo de indicadores}
%     {Derivación}
%     {Controla la operación}
%     \BRitem{Versión}{1.0}
%     \BRitem{Autor}{Sergio Ramírez Camacho}
%     \BRitem{Estatus}{Terminado}
%     \BRitem{Descripción}{
% 			  La información que se utilizará para el cálculo de los indicadores deber considerar los siguientes casos:
% 			  \begin{itemize}
% 			   \item
% 			   \item
% 			  \end{itemize}
% 
%     } 
%     \BRitem{Referenciado por}{
%     \cdtIdRef{CUP 5}{Administrar metas},
%     \cdtIdRef{CUP 7}{Administrar acciones}}
% \end{BusinessRule}
%  
\subsection{Reglas derivadas del negocio}
 
\begin{BusinessRule}{RN-N1}{Nombre de usuario del Coordinador del programa}
    {Restricción}
    {Controla la operación}
    \BRitem{Versión}{1.0}
    \BRitem{Autor}{Natalia Giselle Hernández Sánchez}
    \BRitem{Estatus}{Terminado}
    \BRitem{Descripción}{El nombre de usuario del \cdtRef{actor:usuarioEscuela}{Coordinador del programa} es la \cdtRef{escuela:cct}{clave de centro} de trabajo de la \cdtRef{escuela}{Escuela}}
    \BRitem{Referenciado por}{\cdtIdRef{CUR 19}{Verificar correo electrónico}}%PENDIENTE
\end{BusinessRule}

%------------------------------------------------------------------------------------------------------------------

\begin{BusinessRule}{RN-N2}{Conformación del comité}
	{Derivación}
	{Controla la operación}
	\BRitem{Versión}{1.0}
	\BRitem{Autor}{Victor Lozano Ortega}
	\BRitem{Estatus}{Terminado}
	\item[Descripción:] Un comité debe estar conformado por:
	\begin{itemize}
		\item Un \cdtRef{actor:usuarioEscuela}{Coordinador del programa}.
		\item Un \cdtRef{responsable}{Responsable del programa}.
		\item Uno o dos integrantes por cada línea de acción y al menos uno de ellos debe ser \cdtRef{gls:alumno}{alumno}.
	\end{itemize}
	\BRitem{Referenciado por}{
	\cdtIdRef{CUR 14}{Registrar integrante de línea de acción},
	\cdtIdRef{CUR 15}{Modificar integrante de línea de acción}}
\end{BusinessRule}

%------------------------------------------------------------------------------------------------------------------

\begin{BusinessRule}{RN-N3}{Clave de centro de trabajo válida}
    {Derivación}
    {Controla la operación}
    \BRitem{Versión}{1.0}
    \BRitem{Autor}{Natalia Giselle Hernández Sánchez}
    \BRitem{Estatus}{Terminado}
    \BRitem{Descripción}{Una clave de centro de trabajo válida es aquella que pertenece a una escuela del Estado de México}
    \BRitem{Referenciado por}{
    \cdtIdRef{CUR 3}{Solicitar inscripción}}
\end{BusinessRule}

%------------------------------------------------------------------------------------------------------------------

\begin{BusinessRule}{RN-N4}{Formato de clave de centro de trabajo}
    {Derivación}
    {Controla la operación}
    \BRitem{Versión}{1.0}
    \BRitem{Autor}{Victor Lozano Ortega}
    \BRitem{Estatus}{Terminado}
    \item[Descripción:] La clave de centro de trabajo debe ser una cadena de caracteres con la siguiente estructura ordenada:
    \begin{enumerate}
	\item Un número de dos dígitos que representa la entidad federativa donde se encuentra la escuela.
	\item Una letra mayúscula conocida como \cdtRef{gls:clasificador}{clasificador}.
	\item Dos letras mayúsculas conocidas como \cdtRef{gls:identificadorCCT}{identificador}.
	\item Un número de cuatro dígitos conocido como \cdtRef{gls:numeroProgresivo}{número progresivo}.
	\item Una letra mayúscula conocida como \cdtRef{gls:elementoVerificador}{elemento verificador}.
    \end{enumerate}
    \BRitem{Ejemplo}{15DPR2497K}
    \BRitem{Referenciado por}{
    \cdtIdRef{CUR 3}{Solicitar inscripción}}
\end{BusinessRule}

%------------------------------------------------------------------------------------------------------------------

\begin{BusinessRule}{RN-N5}{Unicidad de integrantes del comité}
    {Restricción}
    {Controla la operación}
    \BRitem{Versión}{1.0}
    \BRitem{Autor}{Natalia Giselle Hernández Sánchez}
    \BRitem{Estatus}{Terminado}
    \BRitem{Descripción}{Una persona no puede estar registrada más de una vez dentro de un comité, esto se verificará con el nombre, el primer apellido, el segundo apellido y la fecha de nacimiento del integrante}
    \BRitem{Referenciado por}{
    \cdtIdRef{CUR 14}{Registrar integrante de línea de acción},
    \cdtIdRef{CUR 15}{Modificar integrante de línea de acción}}
\end{BusinessRule}

%------------------------------------------------------------------------------------------------------------------
\begin{BusinessRule}{RN-N6}{Superficies del predio}
    {Derivación}
    {Controla la operación}
    \BRitem{Versión}{1.0}
    \BRitem{Autor}{Victor Lozano Ortega}
    \BRitem{Estatus}{Terminado}
    \item[Descripción:] Las superficies relacionadas a una escuela son la \cdtRef{escuela:superficieTotal}{superficie total del predio}, 
    y la \cdtRef{escuela:superficieConstruida}{superficie total construida}. 
    La superficie total del predio debe ser mayor o igual a la superficie total construida.
    \BRitem{Referenciado por}{
    \cdtIdRef{CUR 6}{Completar información escolar},
    \cdtIdRef{CUR 7}{Modificar información escolar}} %PENDIENTE
\end{BusinessRule}
%------------------------------------------------------------------------------------------------------------------
\begin{BusinessRule}{RN-N7}{Tiempo para activar una cuenta recién creada}
    {Restricción}
    {Controla la operación}
    \BRitem{Versión}{1.0}
    \BRitem{Autor}{Natalia Giselle Hernández Sánchez}
    \BRitem{Estatus}{Terminado}
    \item[Descripción:] El usuario cuenta con 5 días naturales para activar su cuenta mediante correo electrónico.
    \BRitem{Referenciado por}{
    \cdtIdRef{CUR 19}{Verificar correo electrónico}}
\end{BusinessRule}
%------------------------------------------------------------------------------------------------------------------
\begin{BusinessRule}{RN-N8}{Unicidad de nombres}
	{Restricción de operación}
	{Controla la operación}
	\BRitem{Versión}{1.0}
	\BRitem{Autor}{Jessica Stephany Becerril Delgado}
	\BRitem{Estatus}{Terminado}
	\BRitem{Descripción}{Los nombres científicos de las especies animales y vegetales deben tener nombres que los identifiquen de una manera única en el ámbito en el que se manejen, sin considerar que sean mayúsculas o minúsculas}
	\BRitem{Referenciado por}{
	\cdtIdRef{CUIBB 4}{Registrar fauna}, 
	\cdtIdRef{CUIBB 7}{Registrar flora},
	\cdtIdRef{CUS 13}{Registrar fauna}, 
	\cdtIdRef{CUS 16}{Registrar flora}}
\end{BusinessRule}
%------------------------------------------------------------------------------------------------------------------
\begin{BusinessRule}{RN-N9}{Unicidad de objetivos por línea de acción}
    {Restricción}
    {Controla la operación}
    \BRitem{Versión}{1.0}
    \BRitem{Autor}{Natalia Giselle Hernández Sánchez}
    \BRitem{Estatus}{Terminado}
    \item[Descripción:] Por cada línea de acción se podrá registrar solamente un objetivo.
    \BRitem{Referenciado por}{
    \cdtIdRef{CUP 2}{Registrar objetivo},
    \cdtIdRef{CUP 3}{Modificar objetivo},}
\end{BusinessRule}
%------------------------------------------------------------------------------------------------------------------
\begin{BusinessRule}{RN-N10}{Calcular total de especies}
    {Restricción}
    {Controla la operación}
    \BRitem{Versión}{1.0}
    \BRitem{Autor}{Jessica Stephany Becerril Delgado}
    \BRitem{Estatus}{Terminado}
    \item[Descripción:] El número total de especies es la suma de todas las especies animales o vegetales registradas, según sea el caso, para los inventarios de fauna o flora respectivamente.
    \BRitem{Referenciado por}{
    \cdtIdRef{CUIBB 3}{Administrar inventario de fauna}, 
    \cdtIdRef{CUIBB 6}{Administrar inventario de flora},
    \cdtIdRef{CUS 12}{Administrar inventario de fauna}, 
    \cdtIdRef{CUS 15}{Administrar inventario de flora}}
\end{BusinessRule}
%------------------------------------------------------------------------------------------------------------------
\begin{BusinessRule}{RN-N11}{Calcular total de especies endémicas}
    {Restricción}
    {Controla la operación}
    \BRitem{Versión}{1.0}
    \BRitem{Autor}{Jessica Stephany Becerril Delgado}
    \BRitem{Estatus}{Terminado}
    \item[Descripción:] El número total de especies endémicas es la suma de todas las especies animales o vegetales endémicas, según sea el caso, registradas para los inventarios de fauna o flora respectivamente.
    \BRitem{Referenciado por}{
    \cdtIdRef{CUIBB 3}{Administrar inventario de fauna}, 
    \cdtIdRef{CUIBB 6}{Administrar inventario de flora},
    \cdtIdRef{CUS 12}{Administrar inventario de fauna}, 
    \cdtIdRef{CUS 15}{Administrar inventario de flora}}
\end{BusinessRule}

%------------------------------------------------------------------------------------------------------------------
\begin{BusinessRule}{RN-N12}{Calcular consumo total}
    {Restricción}
    {Controla la operación}
    \BRitem{Versión}{1.0}
    \BRitem{Autor}{Jessica Stephany Becerril Delgado}
    \BRitem{Estatus}{Terminado}
    \item[Descripción:] El consumo total de agua o energía eléctrica, según sea el caso, se calcula sumando el consumo bimestral de cada uno de los recibos ingresados al sistema.
    \BRitem{Referenciado por}{
    \cdtIdRef{CUIBA 2}{Registrar información base para indicadores de agua}, 
    \cdtIdRef{CUIBE 2}{Registrar información base para indicadores de energía},
    \cdtIdRef{CUS 21}{Actualizar información del consumo de agua}, 
    \cdtIdRef{CUS 25}{Actualizar información del cosumo de energía},}
\end{BusinessRule}

%------------------------------------------------------------------------------------------------------------------
\begin{BusinessRule}{RN-N13}{Calcular importe total}
    {Restricción}
    {Controla la operación}
    \BRitem{Versión}{1.0}
    \BRitem{Autor}{Jessica Stephany Becerril Delgado}
    \BRitem{Estatus}{Terminado}
    \item[Descripción:] El importe total de agua o energía eléctrica, según sea el caso, se calcula sumando el importe bimestral de cada uno de los recibos ingresados al sistema.
    \BRitem{Referenciado por}{
    \cdtIdRef{CUIBA 2}{Registrar información base para indicadores de agua}, 
    \cdtIdRef{CUIBE 2}{Registrar información base para indicadores de energía},
    \cdtIdRef{CUS 21}{Actualizar información del consumo de agua}, 
    \cdtIdRef{CUS 25}{Actualizar información del cosumo de energía},}
\end{BusinessRule}
%------------------------------------------------------------------------------------------------------------------
\begin{BusinessRule}{RN-N14}{Reducción de la generación o reciclaje de residuos}
    {Restricción}
    {Controla la operación}
    \BRitem{Versión}{1.0}
    \BRitem{Autor}{Natalia Giselle Hernández Sánchez}
    \BRitem{Estatus}{Terminado}
    \item[Descripción:] Cuando una meta esté enfocada a la reducción de la generación o reciclaje de residuos, es necesario registrar al menos un residuo sólido asociado a la meta.
    \BRitem{Referenciado por}{
    \cdtIdRef{CUPRS 1}{Registrar meta de residuos sólidos}
    \cdtIdRef{CUPRS 2}{Modificar meta de residuos sólidos}}
\end{BusinessRule}
%------------------------------------------------------------------------------------------------------------------
\begin{BusinessRule}{RN-N15}{Restricción de envío del plan de acción}
    {Restricción}
    {Controla la operación}
    \BRitem{Versión}{1.0}
    \BRitem{Autor}{Natalia Giselle Hernández Sánchez}
    \BRitem{Estatus}{Terminado}
    \item[Descripción:] Para el envío del plan de acción, es necesario que exista registrado al menos un objetivo, al menos una meta y al menos una acción.
    \BRitem{Referenciado por}{
    \cdtIdRef{CUP 11}{Enviar plan de acción}}
\end{BusinessRule}

%------------------------------------------------------------------------------------------------------------------
\begin{BusinessRule}{RN-N16}{Calcular avance acumulado}
    {Restricción}
    {Controla la operación}
    \BRitem{Versión}{1.0}
    \BRitem{Autor}{Jessica Stephany Becerril Delgado}
    \BRitem{Estatus}{Terminado}
    \item[Descripción:] El avance acumulado se calcula sumando el valor del avance que se tiene registrado hasta ese momento para la acción o meta y el valor ingresado por el actor al registrar avances.
    \BRitem{Referenciado por}{
    \cdtIdRef{CUS 4}{Registrar avance de meta de ambiente escolar}, \cdtIdRef{CUS 3}{Registrar avance de acciones de ambiente escolar},  \cdtIdRef{CUS 6}{Registrar avance de acciones de consumo responsable}, \cdtIdRef{CUS 7}{Registrar avance de meta de consumo responsable}, \cdtIdRef{CUS 9}{Registrar avance de acciones de biodiversidad},  \cdtIdRef{CUS 10}{Registrar avance de meta de biodiversidad}, \cdtIdRef{CUS 19}{Registrar avance de acciones de agua}, \cdtIdRef{CUS 20}{Registrar avance de meta de agua}, \cdtIdRef{CUS 23}{Registrar avance de acciones de energía}, \cdtIdRef{CUS 24}{Registrar avance de meta de energía}, \cdtIdRef{CUS 27}{Registrar avance de acciones de residuos sólidos}, \cdtIdRef{CUS 28}{Registrar avance de meta de residuos sólidos}}
\end{BusinessRule}

%------------------------------------------------------------------------------------------------------------------
\begin{BusinessRule}{RN-N17}{Calcular avance de la meta}
    {Restricción}
    {Controla la operación}
    \BRitem{Versión}{1.0}
    \BRitem{Autor}{Jessica Stephany Becerril Delgado}
    \BRitem{Estatus}{Terminado}
    \item[Descripción:] El avance de la meta se calcula con la siguiente fórmula $\frac{\textrm{Acciones finalizadas de la meta}}{\textrm{Total de acciones de la meta}}\times 100$
    \BRitem{Referenciado por}{
    \cdtIdRef{CUS 10}{Registrar avance de meta de biodiversidad}, \cdtIdRef{CUS 18}{Administrar avances de agua}, \cdtIdRef{CUS 20}{Registrar avance de meta de agua}, \cdtIdRef{CUS 22}{Administrar avances de energía}, \cdtIdRef{CUS 24}{Registrar avance de meta de energía}, \cdtIdRef{CUS 26}{Administrar avances de residuos sólidos}, \cdtIdRef{CUS 28}{Registrar avance de meta de residuos sólidos}, \cdtIdRef{CUS 4}{Registrar avance de meta de ambiente escolar}, \cdtIdRef{CUS 7}{Registrar avance de meta de consumo responsable}}
\end{BusinessRule}
%------------------------------------------------------------------------------------------------------------------
\begin{BusinessRule}{RN-N18}{Restricción de envío del seguimiento del plan de acción}
    {Restricción}
    {Controla la operación}
    \BRitem{Versión}{1.0}
    \BRitem{Autor}{Natalia Giselle Hernández Sánchez}
    \BRitem{Estatus}{Terminado}
    \item[Descripción:] Para el envío del seguimiento del plan de acción, es necesario que exista registrado al menos el avance de una meta y de una acción.
    \BRitem{Referenciado por}{
    \cdtIdRef{CUS 33}{Enviar seguimiento del plan de acción}}
\end{BusinessRule}