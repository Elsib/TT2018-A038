%============Conclusiones y trabajo a futuro===============

\section{Conclusiones}

Se revisaron las opciones de sensor disponibles y se seleccionó el sensor analógico de tipo fotopletismógrafo debido a que al realizar las pruebas fue con el que se obtuvieron mejores resultados. Al ser un sensor de tipo analógico, se realizó la configuración del periférico del convertidor analógico digital del microcontrolador para tener una frecuencia de muestreo de 128 Hz, obteniendo una señal digital con una resolución de 12 bits.\\

Se revisaron las opciones de sensor disponibles y se seleccionó un sensor digital con uso específico en aplicaciones médicas con interfaz de tipo I2C el cual entrega una señal con una resolución de 16 bits equivalente a 0.00390625$^{\circ}$C, y una precisión de 0.1$^{\circ}$C cuando se realiza la medición entre los 36$^{\circ}$C y 39$^{\circ}$C. Para establecer la comunicación con el microcontrolador, se configuró el periférico I2C a una frecuencia de operación de 400 KHz en el dispositivo maestro. \\

Para establecer la comunicación entre el sistema embebido y el teléfono celular se utiliza una comunicación inalámbrica mediante mensajes de texto. En una primera aproximación se utilizó la red GSM, sin embargo durante el desarrollo del proyecto, las antenas que permiten la comunicación fueron apagadas por lo que se tuvo la necesidad de migrar a las redes UMTS y LTE.\\

Se revisaron opciones del chip de comunicación disponibles y se optó por el módulo 4G que permite establecer comunicación con el microcontrolador mediante UART por lo que se configuró el módulo UART en el microcontrolador a una tasa de transferencia de 19200 baudios.\\

Una vez obtenida la señal de frecuencia cardíaca se aplicaron técnicas de procesamiento digital sobre ella, lo cual consistió en aplicar un offset de 2048 unidades sobre la señal para poder desplazarla y aplicar una ventana de Hamming a 1600 muestras y después el algoritmo de autocorrelación parcial para obtener el espectro de frecuencia de la señal. Sobre el espectro de frecuencia de la señal se aplicó el algoritmo para la búsqueda de la frecuencia fundamental y sobre ese resultado el cálculo de la frecuencia cardíaca, teniendo un error promedio de 1.17\%, de acuerdo a las pruebas realizadas.\\

Para la señal de temperatura debido a que el resultado se guardó en dos registros diferentes se realizó el cálculo del valor real mediante la multiplicación de la parte baja del valor por la resolución del sensor y se sumó a la parte alta de la medición.\\

Finalmente, se configuró el módulo 4G para realizar el envío de mensajes bajo dos condiciones: envío por tiempo cada minuto, siendo este valor es configurable directamente en el código, o al detectar un valor fuera de la ventana definida como normal de 50 a 100 latidos por minuto. De igual manera se tiene una ventana definida como normal de 36$^{\circ}$C a 37$^{\circ}$C. Siendo ambos valores de las ventanas configurables. \\

Al haber obtenido los resultados y cumplir alguna de las condiciones anteriores, se obtiene la fecha y hora proporcionadas por la red 3G o 4G utilizada por el módulo 4G. Con estos valores se generó la trama a enviar mediante el mensaje de texto que contiene la frecuencia cardíaca, temperatura y fecha en la que se obtuvo la medición. \\

También se implementó una aplicación móvil que permitirá realizar el registro de ciertos datos personales de diferentes pacientes con el fin de identificar las mediciones recibidas en el mensaje de texto. Los valores obtenidos del texto del mensaje, son almacenados en una base de datos. Estas mediciones podrán ser consultadas en cualquier momento y se mostrará un promedio de sus mediciones actualizado al momento de almacenar una nueva medición.