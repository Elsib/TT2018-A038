\begin{thebibliography}{0}
	\setlength{\parsep}{0cm}\setlength{\itemsep}{0cm}\setlength{\topsep}{0cm}
	
	\bibitem{ocde2016} OECD (2016), OECD Reviews of Health Systems: Mexico 2016, OECD Publishing, Paris. Disponible en: \url{http://dx.doi.org/10.1787/9789264230491-en}

	\bibitem{who2017} WHO (2017). Las 10 principales causas de defunción. Organización Mundial de la Salud. Disponible en:  \url{http://www.who.int/mediacentre/factsheets/fs310/es/index2.html}
	
	\bibitem{inegiMortalidad} INEGI (s.f.). Mortalidad. Instituto Nacional de Estadística y Geografía. Disponible en: \url{http://www.beta.inegi.org.mx/proyectos/registros/vitales/mortalidad/}
	
	\bibitem{lizarralde2000} Lizarralde, E., Gutiérrez, A. \& Martínez, M. (2000) Alteraciones de la termorregulación. Servicio de Urgencias y Medicina Interna. Hospital de Basurto, Bilbao.

	\bibitem{olvera2013} Olvera, D. \& González J. (2013). Diseño y construcción de un sistema de monitoreo de signos vitales (Tesis de licenciatura). Escuela Superior de Ingeniería Mecánica y Eléctrica, Instituto Politécnico Nacional, México.
	
	\bibitem{guerrero} Guerrero J., López G. \& Ramos, E. Sistema de monitoreo remoto y evaluación de signos vitales en pacientes con enfermedades crónicas. Universidad de Colima, México
	
	\bibitem{ramirez2015} Ramírez M., Martínez D. \& Torres L. (2015). Sistema embebido para monitoreo remoto de signos vitales. TT 2014-B074, 2015, Rubén Ortega González y  Nayeli Vega García.
	
	\bibitem{li2009} C. Li, J. Cummings, J. Lam, E. Graves and W. Wu, "Radar remote monitoring of vital signs," in IEEE Microwave Magazine, vol. 10, no. 1, pp. 47-56, February 2009. Disponible en: \url{http://ieeexplore.ieee.org/stamp/stamp.jsp?tp=\&arnumber=4753999\&isnumber=4753970}
	
	\bibitem{gitbau2011} D. Girbau, A. Ramos, A. Lázaro and R. Villarino, "Remote sensing of vital signs based on Doppler radar and Zigbee interface," 2011 41st European Microwave Conference, Manchester, 2011, pp. 127-130. Disponible en: \url{http://ieeexplore.ieee.org/stamp/stamp.jsp?tp=\&arnumber=6101760\&isnumber=6101646}
	
	\bibitem{cruz2005} D. Cruz and E. Barros, Vital signs remote management system for PDAs, 8th Euromicro Conference on Digital System Design (DSD'05), 2005, pp. 170-173. Disponible en: \url{http://ieeexplore.ieee.org/stamp/stamp.jsp?tp=\&arnumber=1559796\&isnumber=33127}
	
	
	%%%-------Marco Teórico--------%%%
	
%%%%%Signos vitales
	\bibitem{aguayoChile} Aguayo, A \& Lagos A (s.f.) Guía Clínica de Control de Signos Vitales. Universidad Pedro de Valdivia, Chile. Disponible en: \url{http://academico.upv.cl/doctos/KINE-4068/\%7B328B1B37-2C2A-4747-8B38-169806A27753%7D/2012/S1/GUIA%20TECNICA%20DE%20CONTROL%20DE%20SIGNOS%20VITALES%20KINE.pdf}
	
	\bibitem{cobo2011} Cobo, D \& Daza, P (2011). Signos Vitales en Pediatría. Gastrohnup, 13(1), S58-S70. Disponible en: \url{http://bibliotecadigital.univalle.edu.co/bitstream/10893/5810/1/15%20signos.pdf} 
		
	\bibitem{signos2017} Signos Vitales (2017). Universidad Nacional de Mar del Plata, Argentina. 
		
	\bibitem{valoresFreq} Frecuencia cardíaca normal en reposo. Disponible en: \url{https://www.medicalnewstoday.com/articles/291182.php}
	
	\bibitem{HeartRateHarvard} How’s your heart rate and why it matters?. (2017) Disponible en: \url{https://www.health.harvard.edu/heart-health/hows-your-heart-rate-and-why-it-matters}

	\bibitem{FundEspCorFreq} Frecuencia cardíaca. Disponible en: \url{https://fundaciondelcorazon.com/prevencion/riesgo-cardiovascular/frecuencia-cardiaca.html}

	
%	\bibitem{valoresUNAM} Valores normales de signos vitales según la Unidad Interna de Protección Civil del Instituto de Biotecnología de la UNAM. Disponible en: \url{http://www.ibt.unam.mx/server/PRG.base?alterno:0,clase:uipc,pre:cls}
	
	\bibitem{signosvitales2016} Signos vitales (temperatura corporal, pulso, frecuencia respiratoria y presión arterial). (2016). University of Chicago Medicine. Disponible en: \url{http://healthlibrary.uchospitals.edu/Spanish/DiseasesConditions/Adult/NonTraumatic/85,P03963}
	
	\bibitem{tempRCN} Body temperature. (s.f). Royal College of Nursing. Disponible en: \url{https://rcni.com/hosted-content/rcn/first-steps/body-temperature}
	
	\bibitem{talamasSignos} Talamas, J. (s.f.) Habilidades Básicas III, Toma de Signos Vitales. Universidad de Juárez del Estado de Durango, México. Disponible en: \url{http://famen.ujed.mx/doc/manual-de-practicas/a-2016/03_Prac_01.pdf}
	
	\bibitem{proakis} Proakis, J. G., \& Manolakis, D. G. (1992). Digital signal processing: Principles, algorithms, and applications. New York: Macmillan.
	
	\bibitem{signal} Rojas, K., \& Romero, C., \& Romero, P. (2013). Modelo de procesamiento digital de señales cardiacas desarrollado en Matlab. Télématique, 12 (2), 21-36.
	
	\bibitem{sampling} Edmund, L. (2003) Practical Digital Signal Processing, BEng.
	
%%%%%Sistemas embebidos
	\bibitem{vahid1999}	Vahid, F \& Givargis, F. (1999). Embedded System Design: An Unified Hardware/Software Approach. University of California Riverside. Disponible en: \url{https://pdfs.semanticscholar.org/3b62/7703e5b937954ec4637c04dc62637e218166.pdf}
	
	\bibitem{kamal2008} Kamal, R. (2008). Embedded systems: Architecture, programming and design. Boston: McGraw-Hill Higher Education.
	
	\bibitem{delkinEmbSys} Delkin Industrial. (s.f.). Using Embedded Systems in Healthcare. Disponible en:  \url{https://www.delkin.com/blog/using-embedded-systems-in-healthcare/}
	
	\bibitem{nadalEmbebidos} Nadal, A. (s.f.). Sistemas Embebidos. Disponible en:  \url{http://server-die.alc.upv.es/asignaturas/paeees/2005-06/a07%20-%20sistemas%20embebidos.pdf}
	
	\bibitem{heath2005} Heath, S. (2005). Embedded systems design. Oxford: Newnes.
	
%%%%IoT
	\bibitem{vermesanIoT} Vermesan, O \& Friess, P. Internet of Things - From Research and Innovation to Market Deployment. Disponible en:  \url{http://www.internet-of-things-research.eu/pdf/IERC_Cluster_Book_2014_Ch.3_SRIA_WEB.pdf}
	
	\bibitem{laneIoT} N. D. Lane, E. Miluzzo, H. Lu, D. Peebles, T. Choudhury, and A. T.	Campbell, “A survey of mobile phone sensing,” IEEE Communications magazine, vol. 48, no. 9, 2010.
	
	\bibitem{arquitecturaIoT} Sikder, Amit Kumar \& Petracca, Giuseppe \& Aksu, Hidayet \& Jaeger, Trent \& Uluagac, Selcuk. (2018). A Survey on Sensor-based Threats to Internet-of-Things (IoT) Devices and Applications. 
	
%%%%%Redes inalámbricas
	\bibitem{salazarRedes} Salazar, J. Redes Inalámbricas. České vysoké učení technické v Praze Fakulta elektrotechnická. Disponible en: \url{https://upcommons.upc.edu/bitstream/handle/2117/100918/LM01_R_ES.pdf}
	
	\bibitem{camargo2009} Camargo, j.(2009) Modelo de Cobertura para Redes Inalámbricas Interiores. (Tesis de licenciatura). Universidad de Sevilla, Escuela Superior de Ingenieros. Disponible en: \url{http://bibing.us.es/proyectos/abreproy/11761}
	
	\bibitem{yunqyeraWiFi} Yunquera, J. Diseño de una red Wi-Fi para la E.S.I. (Tesis de Licenciatura). Universidad de Sevilla, Escuela Superior de Ingenieros. Disponible en: \url{http://bibing.us.es/proyectos/abreproy/11138/direccion/memoria%252F}
	
	\bibitem{gsm} GSM. Disponible en: \url{https://www.gsma.com/aboutus/gsm-technology/gsm}
	
	\bibitem{2g3g4g} 2G/3G/4G - Is it all about the speed. (2016) Disponible en: \url{https://www.mikroe.com/blog/2g-3g-4g-speed}
	
	\bibitem{wifi} Wifi. Disponible en: \url{https://es.wikipedia.org/wiki/Wifi}
	
	
%%%%Sensores temperatura	
	\bibitem{trerice2001} What is a Temperature Sensor?. (2001). Trerice. Disponible en: \url{http://www.cpinc.com/Trerice/Temperature/63%20-%2064%20temperature.pdf}
	
	\bibitem{maximTemp} Glossary Definition for Temperature Sensor. Maxim integrated. Disponible en: \url{https://www.maximintegrated.com/en/glossary/definitions.mvp/term/Temperature-Sensor/gpk/846}
	
	\bibitem{agarwalTemp} Agarwal, T.  Temperature Sensors - Types, Working \& Operation. Disponible en: \url{https://www.elprocus.com/temperature-sensors-types-working-operation/}
	
	\bibitem{davis2017} Davis, N. (2017). Introduction to Temperature Sensors: Thermistors, Thermocouples, RTDs, and Thermometer ICs. Disponible en: \url{https://www.allaboutcircuits.com/technical-articles/introduction-temperature-sensors-thermistors-thermocouples-thermometer-ic/}
	
	\bibitem{omegaTermopar} Termopar: Tipos y Aplicaciones. Disponible en: \url{ https://mx.omega.com/prodinfo/termopar.html}
	
	\bibitem{elprocusTempSens} Types of Temperature Sensors and Their working Principles. (2016). Disponible en: \url{https://www.elprocus.com/temperature-sensors-types-working-operation/}
	
	\bibitem{amethermTemp} 4 Most Common Types of Temperature Sensor. (2018). Disponible en: \url{https://www.ametherm.com/blog/thermistors/temperature-sensor-types}
	
	\bibitem{omegaTermistor} Introduction to Temperature Measurement with Thermistors. Disponible en: \url{https://www.omega.com/prodinfo/thermistor.html}
	
	\bibitem{unetSensores} Sensores. Universidad Nacional Experimental de Táchira. Disponible en: \url{http://www.unet.edu.ve/~ielectro/sensores.pdf}
	
	\bibitem{omegaCI} Introduction to Integrated Circuit Temperature Sensors. Disponible en: \url{https://www.omega.com/prodinfo/Integrated-Circuit-Sensors.html}
	
	\bibitem{maximThermal} Thermal Management Handbook (2014). Disponible en: \url{https://pdfserv.maximintegrated.com/en/an/AN4679.pdf}
	
%%%%Sensores pulso
	\bibitem{soenh2017} Soenh, A. (2017). Measuring the Heart - How do ECG and PPG Work?. Disponible en: \url{https://imotions.com/blog/measuring-the-heart-how-does-ecg-and-ppg-work/}
	
	\bibitem{naylampECG} Módulo ritmo cardíaco ECG AD8232 con sondas. Disponible en: \url{https://naylampmechatronics.com/biomedico/324-modulo-sensor-de-pulsos-ecg-ad8232-con-sondas.html}
	
	\bibitem{imotionsECG} What is ECG and How Does Id Work?. (2017). Disponible en: \url{https://imotions.com/blog/what-is-ecg/}
	
	\bibitem{salvatore2011} Salvatore, E. (2011). A Brief Look at ECG Sensor Technology. Disponible en: \url{https://www.ecnmag.com/article/2011/08/brief-look-ecg-sensor-technology}
	
	\bibitem{agarwalHS} Agarwal, T. Heartbeat Sensor - Working \& Application. Disponible en: \url{https://www.elprocus.com/heartbeat-sensor-working-application/}
	
%%%%Microcontroladores
	\bibitem{garcia2017} García, V. (2017). Introducción. Trabajo presentado en clase de Introducción a los Microcontroladores, México.
	
%%%%Aplicación móvil
	\bibitem{gardnerApp} Gardner, H \& Davis, K. La Generación App. Como los jóvenes gestionan su identidad, su privacidad y su imaginación. Paidós, México. Disponible en: \url{https://www.popularlibros.com/archivos/9788449329852.pdf}
	
	\bibitem{ibmApp} Desarrollo de aplicaciones móviles nativas, web o híbridas. IBM Software. Disponible en: \url{ftp://ftp.software.ibm.com/la/documents/gb/commons/27754_IBM_WP_Native_Web_or_hybrid_2846853.pdf}
	
	\bibitem{demetrio2013} Demetrio, J. (2013). Web App vs App Nativa. Disponible en: \url{https://www.northware.mx/wp-content/uploads/2013/09/Art%C3%ADculo_Agosto_Northware.pdf}
		
	\bibitem{mobileApps2014} Los 3 Tipos de Aplicaciones Móviles: Ventajas E Inconvenientes. (2014). Disponible en: \url{https://www.lancetalent.com/blog/tipos-de-aplicaciones-moviles-ventajas-inconvenientes/}
	
%-------Diseño de la estructura del sistema--------
	\bibitem{perez2006V} Perez, A.; et al. (2006). Una metodología para el desarrollo de hardware y software embebidos en sistemas críticos de seguridad. Systemics, Cybernetics and Informatics Journal, vol 3, Num. 2, pp. 70-75.

%-------Análisis del módulo de comunicación--------
	\bibitem{garrido2018} Garrido, R. (2018). Bandas y frecuencias en las que trabajan los operadores de México. Disponible en: \url{https://www.xataka.com.mx/telecomunicaciones/estas-son-las-bandas-y-frecuencias-en-las-que-trabajan-los-operadores-de-mexico}
	
%-------Análisis sistemas operativos móviles
	\bibitem{mercadoOS} How to Choose the Right Platform for Mobile App Development. Disponible en: \url{https://steelkiwi.com/blog/how-to-choose-the-right-platform-for-mobile-app-development}
	
	\bibitem{android} Introducción Android OS. Disponible en: \url{https://androidos.readthedocs.io/en/latest/data/introduccion/}
	
	\bibitem{androidJava} Developers (2018). Introducción a Android. Disponible en: \url{https://developer.android.com/guide/}
	
	\bibitem{nations2018iOS} Nations, D. (2018). What Is the iPhone OS (iOS)?. Disponible en: \url{https://www.lifewire.com/what-is-ios-1994355}
	
	\bibitem{swift} Swift. Disponible en: \url{https://www.apple.com/mx/swift/}
	
	
	
\end{thebibliography}