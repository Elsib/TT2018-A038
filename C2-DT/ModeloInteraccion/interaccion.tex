%SUPERFICIES FORESTALES

%----------------------------------------------------------
\section{Entorno de trabajo}

    El entorno de trabajo es el medio por el cual el usuario interactúa con el sistema para poder gestionar la información referente a las Escuelas inscritas en el Programa de Acreditación de Escuelas Ambientalmente Responsables. En este capítulo se describe el comportamiento y los elementos que conforman el entorno de 
    trabajo del PAEAR, como son: la disposición de los elementos principales y comunes de las pantallas, los colores, la iconografía, componentes, etc. \bigskip

    \begin{objetivos}
      \item Describir las áreas principales del entorno de trabajo.
      \item Describir la iconografía utilizada en las pantallas.
     % \item Describir el mapa de navegación del sistema.
      \item Describir los componentes principales de las pantallas, tales como: controles de entrada, datos obligatorios, separadores, tablas de resultados, entre otros.
    \end{objetivos}
\\\\\\\\\\\\\\\\
%----------------------------------------------------------

\cfinput{ModeloInteraccion/interfaces}
\cfinput{ModeloInteraccion/mensajes}