%SUPERFICIES FORESTALES

%----------------------------------------------------------
\section{Entorno de trabajo}

    El entorno de trabajo es el medio por el cual el usuario interactúa con el sistema para poder gestionar la información referente a las Escuelas inscritas en el Programa de Acreditación de Escuelas Ambientalmente Responsables. En este capítulo se describe el comportamiento y los elementos que conforman el entorno de 
    trabajo del PAEAR, como son: la disposición de los elementos principales y comunes de las pantallas, los colores, la iconografía, componentes, etc. \bigskip

    \begin{objetivos}
      \item Describir las áreas principales del entorno de trabajo.
      \item Describir la iconografía utilizada en las pantallas.
      \item Describir el mapa de navegación del sistema.
      \item Describir los componentes principales de las pantallas, tales como: controles de entrada, datos obligatorios, separadores, tablas de resultados, entre otros.
    \end{objetivos}
\\\\\\\\\\\\\\\\
%----------------------------------------------------------

\subsection{Diseño}

  El diseño de las pantallas del sistema sigue un enfoque minimalista que permite a los usuarios trabajar sin gran dificultad y sin distracción. 
  Las pantallas son consistentes, ya que tienen un diseño homogéneo y cuentan con componentes comunes; la consistencia facilita al usuario la interacción
  con el sistema a medida que hace uso del mismo. En la figura~\ref{fig:entornoDeTrabajo} se muestran los elementos principales que conforman las pantallas del sistema, 
  dichos elementos se describen a continuación:

%  \begin{figure}[ht!]
%      \begin{center}
%	  \fbox{\includegraphics[width=.8\textwidth]{images/pantallas/general/layout}}
%	  \caption{Entorno de trabajo del sistema.}
%	  \label{fig:entornoDeTrabajo}
%      \end{center}
%  \end{figure}

    \begin{enumerate}
        \item {\bf Encabezado:} el encabezado tiene la finalidad de mostrar la imagen institucional de la dependencia a la cual pertenece, es decir, la imagen institucional del Gobierno del Estado de México.
        \begin{itemize}
            \item Ancho: $100\%$ del ancho de la ventana del navegador.
            \item Alto: $90px$.
        \end{itemize}

        \item {\bf Menú horizontal:} muestra las opciones generales de navegación para los distintos tipos de usuarios.
        \begin{itemize}
            \item Ancho: $100\%$ del ancho de la pantalla del navegador.
            \item Alto: $40px$.
        \end{itemize}
        
        \item {\bf Menu vertical:} es el área destinada al menú vertical que contendrá los vínculos necesarios para ingresar a las opciones que proporcione el sistema a cada uno de los distintos perfiles de usuarios.
        
        El menú vertical no se encontrará visible para los perfiles de usuario que no requieran del mismo y este espacio será utilizado por el área de trabajo (ver siguiente punto).
        
        \begin{itemize}
            \item Ancho: $20\%$ del ancho de la pantalla del navegador.
            \item Alto: autoajustable al contenido.
        \end{itemize}
        
        \item {\bf Área de trabajo:} en esta sección los usuarios visualizarán los elementos que el sistema proporciona para la realización de las tareas contempladas en el mismo. Aquí se desplegarán formularios para captura, tablas, imágenes, gráficas y demás elementos contenidos en el sistema.\\
        
        El contenido en esta sección se visualizará centrada con base en el ancho y alineado a la parte superior de la misma. Todas las pantallas deberán contar con un título alineado al centro del área de trabajo. 
        \begin{itemize}
            \item Ancho: ancho mínimo $500px$, $65\%$ del ancho de la ventana del navegador web cuando el menu vertical esta visible o el $80\%$ del ancho de la ventana del navegador web en ausencia del menu vertical.
            \item Alto: autoajustable al contenido con un mínimo de $400px$.
        \end{itemize}
        
        \item {\bf Pie:} esta sección contendrá la información de contacto de la unidad correspondiente de la Secretaría del Medio Ambiente del Gobierno del Estado de México.
        \begin{itemize}
            \item Ancho: $80\%$ del ancho de la ventana del navegador.
            \item Alto: $84px$
        \end{itemize}
        
        \item {\bf Información legal:} muestra una leyenda con información legal referente a la propiedad y uso del sistema.
        \begin{itemize}
            \item Ancho: $100\%$ del ancho de la ventana del navegador.
            \item Alto: 24px.
        \end{itemize}
        
        \item {\bf Información de sesión:} esta sección será visible sólo cuando un usuario ingrese al sistema. En ella se mostrarán las opciones para cambiar la contraseña de acceso al mismo, el nombre de usuario y el cierre de sesión.
        \begin{itemize}
            \item Ancho: ajustable al contenido.
            \item Alto: $30px$;
        \end{itemize}
    \end{enumerate}

%----------------------------------------------------------------------


\subsection{Organización}
Las funcionalidades del sistema se encuentra organizadas por menús. Cada actor accede a un menú diferente dependiendo de su perfil, ya que este describe el ciclo de trabajo y las funciones que el actor puede realizar.
%En la parte superior de la pantalla en la interfaz de usuario, se mostrará un menú correspondiente a cada perfil que ingrese al sistema, el cual tendrá las opciones ``Inicio'', ``Información general'', ``Diagnóstico'', ``Plan de acción'' y ``Seguimiento y acreditación'', además de las opciones a las que cada perfil tiene acceso. La opción ``Inicio'' dirige a la pantalla de bienvenida de sistema, mostrada en la figura~\ref{fig:inicio}. La opción ``Información general'' dirige a la pantalla de administración de información escolar  mostrada en la figura~\ref{IUR 5}.  La opción ``Diagnóstico'' dirige a la pantalla de administración de información escolar  mostrada en la figura~\ref{IUD 1}.  La opción ``Plan de acción'' dirige a la pantalla de administración de información escolar  mostrada en la figura~\ref{IUP 1}.  La opción ``Seguimiento y acreditación'' dirige a la pantalla de administración de información escolar  mostrada en la figura~\ref{IUS 5}. A continuación se describen los menús correspondientes a 
cada perfil en el sistema.

\subsubsection{Menú del Coordinador del programa}

    En la figura~\ref{MN2} se muestran las opciones del menú superior que serán visibles para el actor \cdtRef{actor:usuarioEscuela}{Coordinador del programa}. Las opciones del menú se enlistan a continuación:
