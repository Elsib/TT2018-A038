\section{Sistemas embebidos}
	Los sistemas embebidos son combinaciones de software y hardware con restricciones de recursos que se dedica a una aplicación o parte específica de una aplicación o sistema más grande. Son controlados por una computadora incrustada en ellos, lo que implica que se encuentra dentro del sistema general, oculto a la vista, formando una parte integral de un conjunto mayor. Es probable que dicha computadora sea un microprocesador o microcontrolador.\\
	
	Los sistemas embebidos tienen varias características comunes:
	\begin{itemize}
		\item Un sistema embebido generalmente ejecuta solo un programa, repetidamente.
		\item Su diseño debe ser optimizado para reducir costo y espacio. Contienen sólo los recursos de hardware suficientes para cumplir con  los  requerimientos  de  funcionalidad  de  la  aplicación.
		\item Muchos sistemas embebidos deben reaccionar continuamente a los cambios en el entorno del sistema y deben calcular ciertos resultados sin demora.
	\end{itemize}
	\subsection{Arquitectura de los sistemas embebidos}
	\TOCHK{Agregar imagen y descripción de su arquitectura}
	\subsection{Aplicaciones}