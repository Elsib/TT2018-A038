%=================ESTADO DEL ARTE Y MARCO TEÓTICO=================

%=================ESTADO DEL ARTE=================
\section{Estado del arte}
\TODO \textbf{Recuerdo que había que diferenciar entre los artículos y las tesis pero no sé cómo hacerlo. ¿Así está bien?} 
\subsection{Tesis}
	Diana Olvera y José Gonzáles proponen un Sistema de Monitoreo de Signos Vitales capaz de monitorear la presión arterial, el ritmo cardíaco y la temperatura corporal desde el mismo dispositivo con la finalidad de que estas mediciones puedan ser realizadas desde cualquier lugar  sin tener un amplio conocimiento en medicina para su uso. Este sistema se desarrolló implementando el PIC18F4550 para procesar las señales provenientes de los sensores y desplegarlos en un display LCD[2].\\
	
	En el “Sistema de monitoreo remoto y evaluación de signos vitales en pacientes con enfermedades crónicas”, López, Guerrero y Ramos proponen un sistema que sensa los signos vitales de un paciente y utiliza bluetooth y WiFi para transmitir dicha información a un dispositivo móvil encargado de evaluarla y alertar a alguna persona en caso de requerirse[3].\\
	
	Existe un prototipo de hardware/software propuesto por Chávez, Martínez y Torres el cual mediante el uso de un sistema embebido dentro de un microcontrolador DSPIC30F3013 ayuda en el procesamiento, medición y envío mediante una red, tres signos vitales (presión arterial, frecuencia cardiaca y temperatura corporal), dichas mediciones son procesadas por el microcontrolador y enviadas mediante UART hacia un controlador Ethernet para ser recibidas y mostradas por una aplicación diseñada para el usuario[4]. \\

\subsection{Artículos}
	El prototipo diseñado por Li, Cummings, Lan, Graves y Wu en [5], es un sistema de monitoreo de la frecuencia cardiaca y respiratoria de bebés. Está compuesto por una unidad de monitoreo y una unidad receptora. Con un circuito de radiofrecuencia es capaz de producir y recibir señales de radio para la detección de los signos vitales sin la necesidad de que un sensor tenga contacto con el bebé. Utiliza un microcontrolador para procesar la señal recibida y un chip de comunicación XBee para la comunicación inalámbrica con el módulo receptor. \\
	
	En el artículo [6], Girbau, Ramos , Lázaro y Villarino, propone un sistema para el monitoreo de los signos vitales utilizando un radar Doppler y la interfaz Zigbee para enviar la información del sensor de microondas. En este sistema se detecta la respiración y la frecuencia cardiaca, se adquiere con el sistema Zigbee y se transmite vía radiofrecuencia. Permite el monitoreo simultáneo de varias personas localizadas en diferentes nodos Zigbee. \\
	
	Cruz y Barros proponen en [7] el uso de PDAs la adquisición de los signos vitales y su transmisión a un servidor de cuidados de la salud en donde un especialista pueda analizar el electrocardiograma (ECG) generado. Para la adquisición del ECG utiliza electrodos en el paciente. Y para la sincronización de los datos almacenados en la PDA con el servidor, requiere de una conexión a Internet.
	
%=================MARCO TEÓRICO=================
\section{Marco teórico}
	\subsection{IoT}
	\subsection{Sistemas embebidos}
	\subsection{Signos vitales}
		\begin{itemize}
			\item Temperatura
			\item Pulso
			\item Frecuencia respiratoria
			\item Presión arterial
		\end{itemize}
	\subsection{Sensores de temperatura}
		\begin{itemize}
			\item Circuitos integrados
			\item Termistores
			\item Termopares
		\end{itemize}
	\subsection{Sensores de pulso}
		\begin{itemize}
			\item Fotopletismógrafos
			\item ECG
		\end{itemize}
	\subsection{Tecnologías de comunicación}
		\begin{itemize}
			\item Wi-Fi
			\item GSM
			\item ZigBee
		\end{itemize}	