\section{Signos vitales}
	Los signos vitales son mediciones de las funciones más básicas del cuerpo, es decir, indicadores que reflejan el estado físico del paciente y sus órganos vitales, cuyas variaciones expresan cambios ocurridos en el organismo, algunos de índole fisiológico y otros de tipo patológico. \cite{aguayoChile} \cite{cobo2011} 
	Los cuatros signos vitales principales que se examinan de forma rutinaria son los siguientes:
	
	\begin{enumerate}
		\item Frecuencia cardíaca.
		\item Frecuencia respiratoria.
		\item Presión arterial.
		\item Temperatura corporal.
	\end{enumerate}

	Los rangos considerados \textit{normales} de medidas de los signos vitales, varían según la edad, el sexo, el peso, la tolerancia al ejercicio y la enfermedad.
	
	\subsection{Frecuencia cardíaca}
	El pulso está representado por la expansión rítmica de las arterias producida por el pasaje de sangre que es bombeada por el corazón originada en la contracción del ventrículo izquierdo, y que resulta en la expansión y contracción regular de las arterias; representa el rendimiento del latido cardíaco y la adaptación de las arterias.  \cite{aguayoChile} \cite{signos2017} \\
	
	Los valores del pulso arterial se miden a partir de la frecuencia cardíaca, es decir, el número de pulsaciones o latidos que ocurren en un minuto. La frecuencia cardíaca varía dependiendo de diferentes factores, como: la edad, el sexo, la actividad física, la temperatura, entre otros. \cite{valoresFreq}
	
	\begin{table}[htbp]
		\begin{center}
			\begin{tabular}{|l|l|}
				\hline
				\textbf{Edad} & \textbf{Frecuencia cardíaca normal (lpm)} \\
				\hline \hline
				Hasta 1 mes & 70 a 190  \\
				\hline
				De 1 a 11 meses & 80 a 160  \\
				\hline
				De 1 a 2 años & 80 a 130  \\
				\hline
				De 3 a 4 años & 80 a 120  \\
				\hline
				De 5 a 6 años & 75 a 115  \\
				\hline
				De 7 a 10 años & 70 a 110  \\
				\hline
				Más de 10 años & 60 a 100  \\
				\hline
			\end{tabular}
			\caption{Valores normales de frecuencia cardíaca.}
		\end{center}
	\end{table}

	Conocer la frecuencia cardíaca es importante debido a que es un indicador directo de la funcionalidad del corazón. El corazón hace circular oxígeno y sangre rica en nutrientes por todo el cuerpo. Cuando no funciona correctamente, gran parte de las funciones del cuerpo se ven afectadas. \\
	
	Una frecuencia cardíaca lenta, es decir, menor al valor mínimo normal, puede ser un signo de enfermedad, como un ataque cardíaco o alguna otra enfermedad cardíaca, algunas infecciones, altos niveles de potasio en la sangre o una glándula tiroides poco activa, por mencionar algunos. \cite{HeartRateHarvard}\\
	
	Por otro lado, el tener una frecuencia cardíaca rápida, es decir, mayor al valor máximo normal, se asocia a enfermedades como casi cualquier infección causa de fiebre, problemas cardíacos, como la miocardiopatía, algunos medicamentos, niveles bajos de potasio en la sangre, una glándula tiroides hiperactiva, anemia, asma o algún problema respiratorio, entre otros. \cite{HeartRateHarvard}\\
	
	Además, ya sea en personas sanas como en personas hipertensas, con cardiopatía isquémica o con insuficiencia cardíaca, se ha encontrado una relación entre la frecuencia cardíaca y el riesgo de muerte. Ya que mientras mayor sea la frecuencia cardíaca, menor será la expectativa de vida. \cite{FundEspCorFreq}
	
%	\subsection{Frecuencia respiratoria}
%	Respiración es el término que se utiliza para indicar el intercambio de oxígeno y dióxido de carbono que se lleva a cabo en los pulmones y tejidos. \cite{cobo2011} El ciclo respiratorio comprende una fase de inspiración y otra de espiración:
%	\begin{itemize}
%		\item \textbf{Inspiración:} fase activa; se inicia con la contracción del diafragma y los músculos intercostales.
%		\item \textbf{Espiración:} fase pasiva; depende de la elasticidad pulmonar.
%	\end{itemize}
%	
%	La frecuencia respiratoria es el número de respiraciones que suceden en un minuto, y comprende el proceso de inhalación y exhalación. \cite{aguayoChile}\\
%	
%	La frecuencia se mide por lo general cuando una persona está en reposo y consiste simplemente en contar la cantidad de respiraciones durante un minuto cada vez que se eleva el pecho. La frecuencia respiratoria puede aumentar con la fiebre, las enfermedades y otras afecciones médicas. \cite{valoresUNAM}
%	
%	\begin{table}[htbp]
%		\begin{center}
%			\begin{tabular}{|l|l|}
%				\hline
%				\textbf{Edad} & \textbf{Respiraciones por minuto} \\
%				\hline \hline
%				Recién nacido & 30 - 80  \\
%				\hline
%				Lactante menor & 20 - 40  \\
%				\hline
%				Lactante mayor & 20- 30  \\
%				\hline
%				De 2 a 4 años & 20- 30  \\
%				\hline
%				De 6 a 8 años & 20 - 25  \\
%				\hline
%				Adulto & 12 -20  \\
%				\hline
%			\end{tabular}
%			\caption{Valores normales de frecuencia respiratoria.}
%		\end{center}
%	\end{table}
%	
%	\subsection{Presión arterial}
%	Es una medida de la presión que ejerce la sangre sobre las paredes arteriales en su impulso a través de las arterias. Debido a que la sangre se mueve en forma de ondas, existen dos tipos de medidas de presión: la presión sistólica, que es la presión de la sangre debida a la contracción de los ventrículos, es decir, la presión máxima; y la presión diastólica, que es la presión que queda cuando los ventrículos se relajan; ésta es la presión mínima. Tanto la presión sistólica como la diastólica se registran en "mm de Hg" (milímetros de mercurio). \cite{valoresUNAM} \cite{aguayoChile} \cite{signosvitales2016} \\
%	
%	La Presión arterial media se calcula con la siguiente fórmula: 
%	
%	\begin{center}
%		\textit{PA = Presión sistólica – Presión diastólica / 3 + Presión diastólica.}
%	\end{center}
%	
%	\begin{table}[htbp]
%		\begin{center}
%			\begin{tabular}{|l|l|l|}
%				\hline
%				\textbf{Edad} & \textbf{Presión Sistólica (mmHg)} & \textbf{Presión Diastólica (mmHg)} \\
%				\hline \hline
%				Lactante menor & 60 - 90 & 30 - 60 \\
%				\hline
%				2 años  & 78 - 112 & 48 - 78 \\
%				\hline
%				4 años & 85 - 114 & 52 - 85 \\
%				\hline
%				8 años & 95 - 135 & 58 - 88 \\
%				\hline
%				Adulto & 100 - 140 & 60 - 90 \\
%				\hline
%			\end{tabular}
%			\caption{Valores normales de la presión arterial.}
%		\end{center}
%	\end{table}
	
	\subsection{Temperatura corporal}
	La temperatura corporal representa el estado térmico del organismo y expresa el balance entre la producción y pérdida de calor en el cuerpo. \cite{cobo2011} \\
	
	Ya que la temperatura depende de la parte del cuerpo en donde se realice la medición, ya sea interna o externa, por lo que se considera un rango de \textbf{36$^{\circ}$C} a \textbf{37$^{\circ}$C} para la temperatura \textit{normal} del cuerpo humano. \\
	
	Además la temperatura corporal de una persona sana varía a lo largo del día dependiendo del sexo, la actividad reciente, el consumo de alimentos y líquidos, la hora del día y, en el caso de las mujeres, la etapa del ciclo menstrual.  \cite{signosvitales2016} \\
	
%	\begin{table}[htbp]
%		\begin{center}
%			\begin{tabular}{|l|l|}
%				\hline
%				\textbf{Edad} & \textbf{Temperatura corporal (°C)} \\
%				\hline \hline
%				Recién nacido & 36.1 - 37.7  \\
%				\hline
%				Lactante & 37.2  \\
%				\hline
%				De 2 a 8 años & 37.0  \\
%				\hline
%				Adulto & 36.0 - 37.0  \\
%				\hline
%			\end{tabular}
%			\caption{Valores normales de temperatura.}
%		\end{center}
%	\end{table}

	A pesar de todas esas variaciones, el tener un aumento de la temperatura corporal desde el rango normal, podría indicar que la persona tiene una infección. Y una reducción en la temperatura corporal podría indicar el inicio de la hipotermia, lo que puede ser muy grave, particularmente para las personas mayores. \cite{tempRCN}