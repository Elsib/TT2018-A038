	\section{Sensores de temperatura}
	Un sensor de temperatura es un dispositivo que proporciona la medición de temperatura a través de una señal eléctrica. \\
		
	Se utilizan en diversas aplicaciones tales como elaboración de alimentos, climatización para control ambiental, dispositivos médicos, manipulación de productos químicos y control de dispositivos en el sector automotriz, entre otros. \\
		
	Los sensores de temperatura se utilizan para asegurar que la temperatura de algún objeto se encuentre dentro de un cierto rango, lo que proporciona seguridad en el uso de la aplicación. \\
		
	Dependiendo del dispositivo en el que se instalará el sensor de temperatura y el propósito de la aplicación, se deberá utilizar un tipo específico de sensor para medir la temperatura de manera precisa y eficiente, pues en muchos casos la capacidad de respuesta y precisión del circuito de detección puede ser crítica para una pronta decisión.\\
	
	Algunos de los tipos más comunes de sensores de temperatura son los siguientes:
	
		\begin{itemize}
			\item \textbf{Termistores:} Un termistor es un elemento con una resistencia eléctrica que cambia en respuesta a la temperatura. Este nombre se deriva del término más descriptivo "resistencia térmicamente sensible".\\
						
			Los termistores son un tipo de semiconductor, lo que significa que tienen mayor resistencia que los materiales conductores, pero menor resistencia que los materiales aislantes. La relación entre la temperatura de un termistor y su resistencia depende en gran medida de los materiales de los que está compuesta.\\
			
			Existen dos clases de termistores los que presentan un coeficiente negativo de temperatura (NTC), cuya resistencia disminuye con la temperatura y coeficiente positivo con la temperatura  (PTC) cuya resistencia aumenta con la temperatura. Los termistores NTC son los más usados para medición de temperatura.
			
			\item \textbf{Circuitos integrados:}
			\item \textbf{Termopares:} Los termopares se usan comúnmente para medir temperaturas más altas y rangos de temperatura más grandes.\\
			
			Este tipo de sensor de temperatura consta de dos cables de diferentes metales conectados en dos puntos. La tensión variable entre estos dos puntos refleja cambios proporcionales en la temperatura. La precisión es baja, de 0.5 $^{\circ}$C a 5 $^{\circ}$C. Sin embargo, operan en el rango de temperatura más amplio, de -200 $^{\circ}$C a 1750 $^{\circ}$C.			
		\end{itemize}
	
	\section{Sensores de pulso}
		\begin{itemize}
			\item Fotopletismógrafos
			\item ECG
		\end{itemize}
	