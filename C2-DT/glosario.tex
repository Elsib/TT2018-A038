
\begin{itemize}
	\item \textbf{Frecuencia.} Magnitud que mide el número de repeticiones por unidad de tiempo de cualquier fenómeno o suceso periódico.
	\item \textbf{Señal.} Cualquier cantidad física que varía con el tiempo, espacio o cualquier otra variable independiente.
	\item \textbf{ADC}. Convertidor analógico-digital.
	\item \textbf{Muestreo.} Consiste en tomar muestras de una señal analógica a una frecuencia o tasa de muestreo constante, para cuantificarlas posteriormente.
	\item \textbf{Sensor.} Un sensor es un dispositivo que está capacitado para detectar acciones o estímulos externos y responder en consecuencia.
	\item \textbf{Circuito integrado.} Circuito electrónico cuyos componentes, como transmisores y resistencias, están dispuestos en una lámina de material semiconductor.
	\item \textbf{Oscilador.} Un oscilador electrónico es un circuito electrónico que produce una señal electrónica repetitiva, a menudo una onda senoidal o una onda cuadrada.
	\item \textbf{Memoria.} Dispositivo de una máquina donde se almacenan datos o instrucciones que posteriormente se pueden utilizar.
	\item \textbf{SMS.} El servicio de mensajes cortos es un servicio que permite el envío de mensajes de texto entre teléfonos móviles.
	\item \textbf{Aplicación.} Programa preparado para una utilización específica.
	\item \textbf{Interfaz.} Conexión, física o lógica, entre una computadora y el usuario, un dispositivo periférico o un enlace de comunicaciones.
	\item \textbf{SIM.} Una tarjeta SIM es una tarjeta inteligente desmontable usada en teléfonos móviles que almacena de forma segura la clave de servicio del suscriptor usada para identificarse ante la red.
	\item \textbf{Formato Q.} Q es un formato de número de punto fijo donde se especifica el número de bits fraccionarios, y opcionalmente el número de bits de números enteros). Por ejemplo, un número Q15 indica que contiene 15 bits fraccionarios.
	\item \textbf{Baudio.} Unidad de medida de la velocidad de transmisión de señales que se expresa en símbolos por segundo.
\end{itemize}