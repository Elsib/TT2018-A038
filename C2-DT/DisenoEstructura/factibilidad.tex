%========ANÁLISIS DE FACTIBILIDAD
\section{Análisis de Factibilidad}
El estudio de factibilidad se encarga de recopilar datos esenciales del proyecto como la infraestructura tecnológica necesaria, la factibilidad técnica y los recursos económicos necesarios para la implementación del trabajo, esto con el fin de orientar la toma de decisiones que impactarán directamente a la etapa de la ejecución.

\subsection{Factibilidad Técnica}
El propósito  de este análisis es recolectar información sobre los recursos de software, hardware, conocimientos, habilidades y experiencias y la posibilidad de hacer uso de ellos en el desarrollo y la implementación del trabajo, además de definir los requerimientos tecnológicos que deben ser adquiridos para el desarrollo con el menor riesgo posible.\\

A continuación de describen las tecnologías y requerimientos de software y hardware que fueron implementados para el desarrollo del trabajo.

\subsubsection{Software}
Para identificar las diferentes tecnologías de la información que serán usadas para el desarrollo del trabajo, éstas fueron evaluadas y clasificadas en 3 diferentes aspectos.\\

\paragraph{1. Entorno de Desarrollo Integrado.} \textcolor{White}{.} \newline
Para el desarrollo de este trabajo terminal, se hará uso totalmente plataformas libres o en su versión de prueba, con el fin de no necesitar ninguna inversión inicial para su adquisición. En la tabla \ref{disenoEstructura:IDE} se describen los entornos de desarrollo elegidos para trabajar.

\begin{table}[htbp]
	\begin{center}
		\begin{tabular}{|p{3cm}|p{10cm}|}
			\hline
			%			\rowcolor{colorSecundario}
			%			\color{green}
			\thead{Nombre}&\thead{Descripción}\\
			\hline
			\hline
			MPLAB® X IDE &  MPLAB X IDE es un programa de software que se ejecuta en una PC (Windows®, Mac OS®, Linux®) para desarrollar aplicaciones para microcontroladores de microchip y controladores de señales digitales.\\
			\hline
			Android Studio & Android Studio es el entorno de desarrollo integrado (IDE) oficial para el desarrollo de aplicaciones para Android y se basa en IntelliJ IDEA.\\
			\hline
			MATLAB & MATLAB es una herramienta de software matemático que ofrece un entorno de desarrollo integrado (IDE) con un lenguaje de programación propio (lenguaje M). Está disponible para las plataformas Unix, Windows, Mac OS X y GNU/Linux.\\
			\hline
		\end{tabular}
		\caption{Entornos de Desarrollo Integrado a usar}
		\label{disenoEstructura:IDE}
	\end{center}
\end{table}



\paragraph{2. Sistema Operativo}  \textcolor{White}{.} \newline
Puesto que la mayoría de los entornos de desarrollo integrado a usar son multiplataforma, se puede hacer uso de múltiples sistemas operativos para su implementación. En la tabla \ref{disenoEstructura:SO} se muestran las versiones de sistemas operativos que serán evaluados para usar.\\

\begin{table}[htbp]
	\begin{center}
		\begin{tabular}{|p{3cm}|p{10cm}|}
			\hline
			%			\rowcolor{colorSecundario}
			%			\color{green}
			\thead{Nombre}&\thead{Descripción}\\
			\hline
			\hline
			GNU/Linux &  GNU/Linux, también conocido como Linux, es un sistema operativo libre tipo Unix; multiplataforma, multiusuario y multitarea. Es compatible con casi todas las principales plataformas informáticas, incluyendo x86, ARM y SPARC, por lo que es uno de los sistemas operativos más soportados.\\
			\hline
			Windows 8.1® & Es una actualización de Windows 8®. Presenta una expansión de configuración de PC para incluir más opciones previamente exclusivas del  Panel de control de Windows®. Tiene la capacidad de sincronizar más configuraciones entre dispositivos, incluyendo configuraciones de la pantalla de Inicio y configuración de teclado Bluetooth y ratón.\\
			\hline
			Windows 10® & Windows 10® es el último sistema operativo desarrollado por Microsoft®  como parte de la familia de sistemas operativos Windows NT. Uno de los aspectos más importantes de Windows 10 es el enfoque en la armonización de experiencias de usuario y funcionalidad entre diferentes tipos de dispositivos, además de abordar las deficiencias en la interfaz de usuario de Windows®  que se introdujo por primera vez en Windows 8®.\\
			\hline
		\end{tabular}
		\caption{Sistemas operativos}
		\label{disenoEstructura:SO}
	\end{center}
\end{table}

Después de este análisis, se determinó que se puede hacer uso de más de un sistema operativo, por lo que en cada uno de los equipos se tendrá instalado el sistema operativo Linux en su distribución Zorin OS 12.4 64-bit y Windows 10®.

\paragraph{3. Lenguaje de Programación} \textcolor{White}{.} \newline
En la tabla \ref{disenoEstructura:lenguajes} se muestran describen los lenguajes de programación que serán usados.\\


\begin{table}[htbp]
	\begin{center}
		\begin{tabular}{|p{3cm}|p{12cm}|}
			\hline
			%			\rowcolor{colorSecundario}
			%			\color{green}
			\thead{Nombre}&\thead{Descripción}\\
			\hline
			\hline
			C &  C es un lenguaje de programación de tipos de datos estáticos, débilmente tipificado, de medio nivel, ya que dispone de las estructuras típicas de los lenguajes de alto nivel pero, a su vez, dispone de construcciones del lenguaje que permiten un control a muy bajo nivel. Los compiladores suelen ofrecer extensiones al lenguaje que posibilitan mezclar código en ensamblador con código C o acceder directamente a memoria o dispositivos periféricos.\\
			\hline
			Ensamblador & El lenguaje ensamblador es un lenguaje de programación de bajo nivel. Consiste en un conjunto de mnemónicos que representan instrucciones básicas para los computadores, microprocesadores, microcontroladores y otros circuitos integrados programables. Implementa una representación simbólica de los códigos de máquina binarios y otras constantes necesarias para programar una arquitectura de procesador y constituye la representación más directa del código máquina específico para cada arquitectura legible por un programador.\\
			\hline
			Java & Java es un lenguaje de programación de propósito general, concurrente, orientado a objetos, que fue diseñado específicamente para tener tan pocas dependencias de implementación como fuera posible. Su intención es permitir que los desarrolladores de aplicaciones escriban el programa una vez y lo ejecuten en cualquier dispositivo, lo que quiere decir que el código que es ejecutado en una plataforma no tiene que ser recompilado para correr en otra.
			Java fue elegido como el lenguaje para el entorno de desarrollo de Android.\\
			\hline
			M & Las aplicaciones de MATLAB se desarrollan en un lenguaje de programación propio. Este lenguaje es interpretado, y puede ejecutarse tanto en el entorno interactivo, como a través de un archivo de script (archivos *.m). Este lenguaje permite operaciones de vectores y matrices, funciones, cálculo lambda, y programación orientada a objetos.\\
			\hline
		\end{tabular}
		\caption{Lenguajes de programación}
		\label{disenoEstructura:lenguajes}
	\end{center}
\end{table}

Debido a la naturaleza del presente trabajo terminal, es necesario el uso de más de un lenguaje de programación. Para la programación del microcontrolador, se hará uso de los lenguajes C y Ensamblador, los cuales son soportados por el entorno de desarrollo integrado MPLAB X. Adicionalmente, se hará uso del lenguaje java, implementado para el desarrollo de la aplicación móvil a través de Android Studio. Y para la ejecución de las pruebas unitarias, el lenguaje utilizado será M, en el IDE de MATLAB.


%\paragraph{4. Gestor de Base de Datos} \textcolor{White}{.} \newline

\subsubsection{Hardware}
Evaluando el equipo de cómputo con el que se cuenta y analizando las especificaciones y requisitos de los entornos de desarrollo elegidos, no es necesario realizar una inversión inicial para la adquisición de nuevos equipos pues las características con las que cuentan, satisfacen los requerimientos necesarios para el desarrollo del trabajo. Sin embargo, para el hardware que será implementado en el sistema embebido, sí es necesario realizar una inversión inicial por el costo unitario de cada uno.\\

En la tabla \ref{disenoEstructura:equipos} se describen las especificaciones de los equipos que se tienen y que serán utilizados para el diseño, desarrollo e implementación del trabajo.

\begin{table}[htbp]
	\begin{center}
		\begin{tabular}{|c|p{5cm}|p{7cm}|}
			\hline
			%			\rowcolor{colorSecundario}
			%			\color{green}
			Cantidad&Equipo&Características\\
			\hline
			\hline
			1 & Sony VAIO Pro SVP132A1CU & \begin{UClist}
				\UCli Procesador: Intel® Core i5-4200U CPU $@$ 1.60 GHz x 4
				\UCli RAM: 8GB
				\UCli Sistema Operativo: Windows 10 64-bit / Zorin OS 64bit
				\end{UClist} \\
			\hline
			1 & Asus VivoBook S510U & \begin{UClist}
				\UCli Procesador: Intel® Core i7-8550U CPU $@$ 1.80 GHz x 8
				\UCli RAM: 8GB
				\UCli Sistema Operativo: Windows 10 64-bit / Zorin OS 64bit
			\end{UClist}\\
			\hline 
			1 & Smartphone Motorola G3 & \begin{UClist}
				\UCli Procesador: Snapdragon 410
				\UCli RAM: 1GB
				\UCli Sistema Operativo: Android 6.0 Mashmallow
			\end{UClist}\\
			\hline
			1 & Smartphone Xiaomi Mi A2 &\begin{UClist}
				\UCli Procesador: Snapdragon 660
				\UCli RAM: 4GB
				\UCli Sistema Operativo: Android 8.1 Oreo con Android One
			\end{UClist}\\
			\hline
		\end{tabular}
		\caption{Características del equipo disponible}
		\label{disenoEstructura:equipos}
	\end{center}
\end{table}

En la tabla \ref{disenoEstructura:recursosHardware} se listan los recursos de hardware que, según el análisis realizado en el capítulo \ref{chp:analisis} es necesario adquirir para el diseño, desarrollo e implementación del trabajo.\\

\begin{table}[htbp]
	\begin{center}
		\begin{tabular}{|c|p{10cm}|}
			\hline
			%			\rowcolor{colorSecundario}
			%			\color{green}
			\thead{Cantidad}&\thead{Recurso}\\
			\hline
			\hline
			1 & Microcontrolador dsPIC30F4013 \\
			\hline
			1 & Módulo 4G \\
			\hline
			1 & SIM telefónico para módulo GSM\\
			\hline
			1 & Sensor de temperatura MAX30205\\
			\hline
			1 & Sensor de pulso\\
			\hline
			1 & Programador para microcontrolador\\
			\hline
			1 & Módulo FT232 \\
			\hline
		\end{tabular}
		\caption{Recursos de hardware necesarios}
		\label{disenoEstructura:recursosHardware}
	\end{center}
\end{table}

Como resultado del análisis de factibilidad técnica, se determinó que se cuenta con la infraestructura tecnológica necesaria para el desarrollo del trabajo terminal.

%==============================================ECONÓMICA==================================================

\subsection{Factibilidad Económica}
El análisis de la factibilidad económica determina si los recursos económicos y financieros  son suficientes para llevar a cabo las actividades o procesos, además permite conocer los costos estimados para el desarrollo del trabajo.\\

Para determinar la factibilidad económica del presente trabajo terminal se realizó un análisis describiendo los gastos totales, los cuales fueron clasificados en las siguientes categorías:

\begin{enumerate}
	\item Gastos tecnológicos
	\item Gastos por servicios
\end{enumerate}

Debido a este proyecto tiene fines académicos, en los gastos descritos no se consideran sueldos.


\paragraph{1. Gastos tecnológicos} \textcolor{White}{.} \newline
En esta sección se consideran todos los gastos relacionados con el software y hardware implementado en el desarrollo del proyecto. Todas las cantidades descritas están consideradas en pesos mexicanos.\\

En la tabla \ref{disenoEstructura:gastosDepreciacion} se describen los gastos del equipo que fue utilizado pero que no se realizó una inversión inicial pues ya se contaba con él. Para estos equipos se consideró únicamente el gasto de la depreciación durante los meses de trabajo. El porcentaje de depreciación anual que se consideró para todos los equipos fue del 20\% y el tiempo de trabajo en meses estimado para este trabajo terminal fue de 9.

\begin{table}[htbp]
	\begin{center}
		\begin{tabular}{|p{4cm}|p{3cm}|p{3cm}|p{3cm}|}
			\hline
			%			\rowcolor{colorSecundario}
			%			\color{green}
			\thead{Equipo}&\thead{Precio de compra\\(\$)}&\thead{Depreciación mensual\\(\$)}&\thead{Depreciación total\\(\$)} \\
			\hline
			\hline
			Sony VAIO Pro SVP132A1CU &15,000.00 &250.00&2,250.00 \\
			\hline
			Asus VivoBook S510U & 24,000.00&400.00&3,600.00 \\
			\hline
			Smartphone Motorola G3 &3,000.00 &50.00&450.00\\
			\hline
			Smartphone Xiaomi Mi A2 & 5,000.00&83.33&750.00\\
			\hline
			\hline
			TOTAL & &&7,050.00\\
			\hline
		\end{tabular}
		\caption{Gastos tecnológicos por depreciación}
		\label{disenoEstructura:gastosDepreciacion}
	\end{center}
\end{table}

En la tabla \ref{disenoEstructura:gastosHardware} se describen los gastos relacionados con la compra del equipo de hardware específico para el desarrollo del sistema embebido.

\begin{table}[htbp]
	\begin{center}
		\begin{tabular}{|c|p{4cm}|p{3cm}|p{3cm}|}
			\hline
			%			\rowcolor{colorSecundario}
			%			\color{green}
			\thead{Cantidad}&\thead{Recurso}&\thead{Precio unitario\\(\$)}&\thead{Subtotal\\(\$)} \\
			\hline
			\hline
			1 &Microcontrolador dsPIC30F4013&143.39&143.39\\
			\hline
			1 & Módulo GSM&1,072.00&1,072.00 \\
			\hline
			1 &SIM telefónico para módulo GSM&0.00&0.00\\
			\hline
			1 &Sensor de temperatura MAX30205&30.57&30.57\\
			\hline
			1 &Sensor de pulso&477.46&477.46\\
			\hline
			1 &Programador para microcontrolador&300.00&300.00\\
			\hline
			1 &Módulo FT232&100.00&100.00\\
			\hline
			\hline
			TOTAL & &&2,123.42\\
			\hline
		\end{tabular}
		\caption{Gastos tecnológicos por hardware}
		\label{disenoEstructura:gastosHardware}
	\end{center}
\end{table}


\paragraph{2. Gastos por servicios} \textcolor{White}{.} \newline
El desarrollo del trabajo implica gastos para los servicios con los que funcionarán los recursos de hardware y software mencionados anteriormente. El gasto estimado por los servicios a utilizar se describe en la tabla \ref{disenoEstructura:gastosServicios}.\\


\begin{table}[htbp]
	\begin{center}
		\begin{tabular}{|p{4cm}|p{3cm}|p{3cm}|}
			\hline
			%			\rowcolor{colorSecundario}
			%			\color{green}
			\thead{Servicio}&\thead{Gasto mensual\\(\$)}&\thead{Gasto total\\(\$)}\\
			\hline
			\hline
			Internet &150.00 &1,350.00 \\
			\hline
			Luz eléctrica &135.00 &1,215.00 \\
			\hline
			SIM telefónica &50.00 &450.00\\
			\hline
			\hline
			TOTAL & &3,015\\
			\hline
		\end{tabular}
		\caption{Gastos por servicios}
		\label{disenoEstructura:gastosServicios}
	\end{center}
\end{table}

\paragraph{Gastos totales} \textcolor{White}{.} \newline
Para obtener el monto total de los gastos para el proyecto, se sumaron los gastos mencionados anteriormente, como se muestra en la tabla \ref{disenoEstructura:gastosTotales}, por lo tanto el costo total estimado del proyecto es: \$9,176  pesos mexicanos.

\begin{table}[htbp]
	\begin{center}
		\begin{tabular}{|p{4cm}|p{4cm}|}
			\hline
			%			\rowcolor{colorSecundario}
			%			\color{green}
			\thead{Concepto}&\thead{Gasto\\(\$)}\\
			\hline
			\hline
			Gastos tecnológicos &  7,050.00\\
			\hline
			Gastos por servicios & 3,015.00\\
			\hline
			\hline
			TOTAL & 9,176.43\\
			\hline
		\end{tabular}
		\caption{Gastos totales}
		\label{disenoEstructura:gastosTotales}
	\end{center}
\end{table}
  

