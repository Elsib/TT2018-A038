%========ANÁLISIS DE FACTIBILIDAD
\section{Análisis de Factibilidad}
El estudio de factibilidad se encarga de recopilar datos esenciales del proyecto como la infraestructura tecnológica necesaria, la factibilidad técnica y los recursos económicos necesarios para la implementación del trabajo, esto con el fin de orientar la toma de decisiones que impactarán directamente a la etapa de la ejecución.

\subsection{Factibilidad Técnica}
El propósito  de este análisis es recolectar información sobre los recursos de software, hardware, conocimientos, habilidades y experiencias y la posibilidad de hacer uso de ellos en el desarrollo y la implementación del trabajo, además de definir los requerimientos tecnológicos que deben ser adquiridos para el desarrollo con el menor riesgo posible.\\

A continuación de describen las tecnologías y requerimientos de software y hardware que fueron implementados para el desarrollo del trabajo.

\subsubsection{Software}
Para identificar las diferentes tecnologías de la información que serán usadas para el desarrollo del trabajo, éstas fueron evaluadas y clasificadas en 4 diferentes aspectos.\\

\paragraph{1. Entorno de Desarrollo Integrado.}
Para el desarrollo de este trabajo terminal, se hará uso totalmente plataformas libres o en su versión de prueba, con el fin de no necesitar ninguna inversión inicial para su adquisición. En la tabla \ref{disenoEstructura:IDE} se describen los entornos de desarrollo elegidos para trabajar.

\begin{table}[htbp]
	\begin{center}
		\begin{tabular}{|p{3cm}|p{10cm}|}
			\hline
			%			\rowcolor{colorSecundario}
			%			\color{green}
			\thead{Nombre}&\thead{Descripción}\\
			\hline
			\hline
			MPLAB® X IDE &  MPLAB X IDE es un programa de software que se ejecuta en una PC (Windows®, Mac OS®, Linux®) para desarrollar aplicaciones para microcontroladores de microchip y controladores de señales digitales.\\
			\hline
			Android Studio & Android Studio es el entorno de desarrollo integrado (IDE) oficial para el desarrollo de aplicaciones para Android y se basa en IntelliJ IDEA.\\
			\hline
			MATLAB & MATLAB es una herramienta de software matemático que ofrece un entorno de desarrollo integrado (IDE) con un lenguaje de programación propio (lenguaje M). Está disponible para las plataformas Unix, Windows, Mac OS X y GNU/Linux.\\
			\hline
		\end{tabular}
		\caption{Entornos de Desarrollo Integrado a usar}
		\label{disenoEstructura:IDE}
	\end{center}
\end{table}

\paragraph{2. Sistema Operativo.}
Puesto que la mayoría de los entornos de desarrollo integrado a usar son multiplataforma, se puede hacer uso de múltiples sistemas operativos para su implementación. En la tabla \ref{disenoEstructura:SO} se muestran las versiones de sistemas operativos que serán evaluados para usar.\\

\begin{table}[htbp]
	\begin{center}
		\begin{tabular}{|p{3cm}|p{10cm}|}
			\hline
			%			\rowcolor{colorSecundario}
			%			\color{green}
			\thead{Nombre}&\thead{Descripción}\\
			\hline
			\hline
			GNU/Linux &  GNU/Linux, también conocido como Linux, es un sistema operativo libre tipo Unix; multiplataforma, multiusuario y multitarea. Es compatible con casi todas las principales plataformas informáticas, incluyendo x86, ARM y SPARC, por lo que es uno de los sistemas operativos más soportados.\\
			\hline
			Windows 8.1® & Es una actualización de Windows 8®. Presenta una expansión de configuración de PC para incluir más opciones previamente exclusivas del  Panel de control de Windows®. Tiene la capacidad de sincronizar más configuraciones entre dispositivos, incluyendo configuraciones de la pantalla de Inicio y configuración de teclado Bluetooth y ratón.\\
			\hline
			Windows 10® & Windows 10® es el último sistema operativo desarrollado por Microsoft®  como parte de la familia de sistemas operativos Windows NT. Uno de los aspectos más importantes de Windows 10 es el enfoque en la armonización de experiencias de usuario y funcionalidad entre diferentes tipos de dispositivos, además de abordar las deficiencias en la interfaz de usuario de Windows®  que se introdujo por primera vez en Windows 8®.\\
			\hline
		\end{tabular}
		\caption{Sistemas operativos}
		\label{disenoEstructura:SO}
	\end{center}
\end{table}

Después de este análisis, se determinó que se puede hacer uso de más de un sistema operativo, por lo que en cada uno de los equipos se tendrá instalado el sistema operativo Linux en su distribución Zorin OS 12.4 64-bit y Windows 10®.

\paragraph{3. Lenguaje de Programación}
En la tabla \ref{disenoEstructura:lenguajes} se muestran describen los lenguajes de programación que serán usados dependiendo de \\


\begin{table}[htbp]
	\begin{center}
		\begin{tabular}{|p{3cm}|p{12cm}|}
			\hline
			%			\rowcolor{colorSecundario}
			%			\color{green}
			\thead{Nombre}&\thead{Descripción}\\
			\hline
			\hline
			C &  C es un lenguaje de programación de tipos de datos estáticos, débilmente tipificado, de medio nivel, ya que dispone de las estructuras típicas de los lenguajes de alto nivel pero, a su vez, dispone de construcciones del lenguaje que permiten un control a muy bajo nivel. Los compiladores suelen ofrecer extensiones al lenguaje que posibilitan mezclar código en ensamblador con código C o acceder directamente a memoria o dispositivos periféricos.\\
			\hline
			Ensamblador & El lenguaje ensamblador es un lenguaje de programación de bajo nivel. Consiste en un conjunto de mnemónicos que representan instrucciones básicas para los computadores, microprocesadores, microcontroladores y otros circuitos integrados programables. Implementa una representación simbólica de los códigos de máquina binarios y otras constantes necesarias para programar una arquitectura de procesador y constituye la representación más directa del código máquina específico para cada arquitectura legible por un programador.\\
			\hline
			Java & Java es un lenguaje de programación de propósito general, concurrente, orientado a objetos, que fue diseñado específicamente para tener tan pocas dependencias de implementación como fuera posible. Su intención es permitir que los desarrolladores de aplicaciones escriban el programa una vez y lo ejecuten en cualquier dispositivo, lo que quiere decir que el código que es ejecutado en una plataforma no tiene que ser recompilado para correr en otra.
			Java fue elegido como el lenguaje para el entorno de desarrollo de Android.\\
			\hline
			M & Las aplicaciones de MATLAB se desarrollan en un lenguaje de programación propio. Este lenguaje es interpretado, y puede ejecutarse tanto en el entorno interactivo, como a través de un archivo de script (archivos *.m). Este lenguaje permite operaciones de vectores y matrices, funciones, cálculo lambda, y programación orientada a objetos.\\
			\hline
		\end{tabular}
		\caption{Lenguajes de programación}
		\label{disenoEstructura:lenguajes}
	\end{center}
\end{table}

Debido a la naturaleza del presente trabajo terminal, es necesario el uso de más de un lenguaje de programación. Para la programación del microcontrolador, se hará uso de los lenguajes C y Ensamblador, los cuales son soportados por el entorno de desarrollo integrado MPLAB X. Adicionalmente, se hará uso del lenguaje java, implementado para el desarrollo de la aplicación móvil a través de Android Studio. Y para la ejecución de las pruebas unitarias, el lenguaje utilizado será M, en el IDE de MATLAB.


\paragraph{4. Gestor de Base de Datos}


